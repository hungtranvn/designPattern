\documentclass[13pt]{beamer}
%
% Choose how your presentation looks.
%
% For more themes, color themes and font themes, see:
% http://deic.uab.es/~iblanes/beamer_gallery/index_by_theme.html
%
\mode<presentation>
{
  \usetheme{CambridgeUS}     % or try Darmstadt, Madrid, Warsaw, ...
  \usecolortheme{beaver} % or try albatross, beaver, crane, ...
  \usefonttheme{default}  % or try serif, structurebold, ...
  \setbeamertemplate{navigation symbols}{}
  \setbeamertemplate{caption}[numbered]
} 

\usepackage[english]{babel}
\usepackage[utf8x]{inputenc}
\usepackage{xcolor}
\usepackage{multicol}
\usepackage{tikz}
\usepackage{tikz-uml}
\tikzumlset{font=\footnotesize\ttfamily, class width=6ex}
\usepackage{hyperref}

\usepackage{listings}
\definecolor{codegreen}{rgb}{0,0.6,0}
\definecolor{codegray}{rgb}{0.5,0.5,0.5}
\definecolor{codepurple}{rgb}{0.58,0,0.82}
\definecolor{backcolour}{rgb}{0.95,0.95,0.92}

\lstdefinestyle{myCustomCppStyle}{
  language=C++,
  numbers=left,
  stepnumber=1,
  numbersep=9pt,
  tabsize=2,
  showspaces=false,
  showstringspaces=false
}

\lstset{basicstyle=\tiny,style=myCustomCppStyle}

\lstdefinestyle{mystyle}{
    backgroundcolor=\color{backcolour},   
    commentstyle=\color{codegreen},
    keywordstyle=\color{magenta},
    numberstyle=\tiny\color{codegray},
    stringstyle=\color{codepurple},
    basicstyle=\ttfamily\footnotesize,
    breakatwhitespace=false,         
    breaklines=true,                 
    captionpos=b,                    
    keepspaces=true,                 
    numbers=left,                    
    numbersep=5pt,                  
    showspaces=false,                
    showstringspaces=false,
    showtabs=false,                  
    tabsize=1
}

\lstset{style=mystyle}

\usepackage{graphicx}
\graphicspath{ {./images/} }

\usepackage{tikz}
\usetikzlibrary{decorations.text}
\usetikzlibrary{shapes.geometric, arrows, positioning, calc, matrix}

\tikzset{
  basic box/.style={
    shape=rectangle, rounded corners, align=center,
    draw=#1, fill=#1!25},
  header node/.style={
    Minimum Width=header nodes,
    font=\strut\Large\ttfamily,
    text depth=+0pt,
    fill=white, draw},
  header/.style={%
    inner ysep=+1.5em,
    append after command={
      \pgfextra{\let\TikZlastnode\tikzlastnode}
      node [header node] (header-\TikZlastnode) at (\TikZlastnode.north) {#1}
      node [span=(\TikZlastnode)(header-\TikZlastnode)] at (fit bounding box) (h-\TikZlastnode) {}
    }
  },
  hv/.style={to path={-|(\tikztotarget)\tikztonodes}},
  vh/.style={to path={|-(\tikztotarget)\tikztonodes}},
  fat blue line/.style={ultra thick, blue}
}

\definecolor{mygray}{RGB}{208,208,208}
\definecolor{mymagenta}{RGB}{226,0,116}
\newcommand*{\mytextstyle}{\sffamily\Large\bfseries\color{black!85}}
\newcommand{\arcarrow}[3]{%
   % inner radius, middle radius, outer radius, start angle,
   % end angle, tip protusion angle, options, text
   \pgfmathsetmacro{\rin}{1.7}
   \pgfmathsetmacro{\rmid}{2.2}
   \pgfmathsetmacro{\rout}{2.7}
   \pgfmathsetmacro{\astart}{#1}
   \pgfmathsetmacro{\aend}{#2}
   \pgfmathsetmacro{\atip}{5}
   \fill[mygray, very thick] (\astart+\atip:\rin)
                         arc (\astart+\atip:\aend:\rin)
      -- (\aend-\atip:\rmid)
      -- (\aend:\rout)   arc (\aend:\astart+\atip:\rout)
      -- (\astart:\rmid) -- cycle;
   \path[
      decoration = {
         text along path,
         text = {|\mytextstyle|#3},
         text align = {align = center},
         raise = -1.0ex
      },
      decorate
   ](\astart+\atip:\rmid) arc (\astart+\atip:\aend+\atip:\rmid);
}
\title[Design Pattern]{Creational Design Pattern}
\author{Hung Tran}
\institute{Fpt software}
\date{\today}

\begin{document}

\begin{frame}
  \titlepage
\end{frame}

% Uncomment these lines for an automatically generated outline.
\begin{frame}{Outline}
  \tableofcontents
\end{frame}

\section{Creational Pattern Overview}

\begin{frame}{Creational Pattern Overview}
	\begin{center}
	\textcolor{blue}{\textbf{Construction process of an object.}}
	\end{center}
	\begin{itemize}
		\setlength\itemsep{1em}
		\item \textbf{Singleton}: Ensure only one instance.
		\item \textbf{Factory Method}: Create instance without depending on its concrete type.
		\item \textbf{Object pool}: Reuse existing instances.
		\item \textbf{Abstract factory}: Create instances from a specific family.
		\item \textbf{Prototype}: Clone existing objects from a prototype.
		\item \textbf{Builder}: Construct a complex object step by step.
	\end{itemize}
\end{frame}

\section{Object Pool Design Pattern}

\begin{frame}{Game Example}
\end{frame}

\begin{frame}{Factory Pattern}
\begin{itemize}
\item Always create new instance
\end{itemize}
\end{frame}

\begin{frame}{The Intent of Object Pool Design Pattern}
	\begin{center}
	\textcolor{blue}{\textbf{Improve performance and memory use by reusing objects from fixed pool instead of allocating and freeing them repetitively.}}
	\end{center}
\end{frame}

\begin{frame}{Structure of Object Pool Design Pattern}
	\begin{center}
	\begin{tikzpicture}
	\umlemptyclass[x=0,y=0]{Clients}
	\umlclass[x=8,y=0]{ObjectPool}{pool: vector}{GetInstance() \\ Acquire() \\ Release()}
	\umlclass[x=8,y=-4]{SharedObject}{}{MethodA() \\ MethodB()}
	\umluniassoc[pos=0.95, align=right, name=uniassoc,]{ObjectPool}{SharedObject}
	\umluniassoc[pos=0.95, align=right, name=uniassoc, arg1=requests]{Clients}{ObjectPool}
	\umluniassoc[geometry = |-,pos=0.95, align=right, name=uniassoc, arg1=uses]{Clients}{SharedObject}
	\end{tikzpicture}	
	\end{center}
\end{frame}

\begin{frame}{How to Implement of Object Pool Design Pattern?}
	\begin{itemize}
		\setlength\itemsep{1em}
		\item \emph{ObjectPool} class maintains an array or a list of \emph{SharedObject} instances
		\item \emph{SharedObject} instances are not created by Clients; instead they use the \emph{ObjectPool} class
		\item Objects are constructed when: The program starts, The pool is empty, An existing object is not available.
		\item For the last case, the pool can be \emph{grow dynamically}.
		\item ObjectPool should have \emph{only one instance} (Singleton or Monostate)
	\end{itemize}
\end{frame}

\begin{frame}{How to Implement of Object Pool Design Pattern?}
	\begin{itemize}
		\setlength\itemsep{1em}
		\item The clients acquire a \emph{SharedObject} instance by invoking a factory method in the pool.
		\item When the clients gets a \emph{SharedObject} instance, it is either removed from the \emph{ObjectPool} or marked as used.
		\item The clients may manually return a \emph{SharedObject} to the \emph{ObjectPool} or t may be done automatically.
		\item The instance can be used again.
	\end{itemize}
\end{frame}

\begin{frame}{How to Implement of Object Pool Design Pattern?}
	\begin{itemize}
		\setlength\itemsep{1em}
		\item The Pool object instance can be reset: before giving it to the client or after it is returned to the pool.
		\item The ObjectPool is responsible for deleting the pooled instances
		\item These instances are usually deleted at the end of the program.
		\item To avoid tight coupling with concreate pooled objects, ObjectPool can use a factory to instantiate them.
	\end{itemize}
\end{frame}

\begin{frame}{Basic Implementation}
\begin{columns}[T]
\begin{column}{.45\textwidth}
\lstinputlisting{./examples/basic/SharedObject.h}
\end{column}

\begin{column}{.45\textwidth}
\lstinputlisting{./examples/basic/SharedObject.cpp}
\end{column}
\end{columns}
\end{frame}

\begin{frame}{Basic Implementation}
\begin{columns}[T]
\begin{column}{.45\textwidth}
\lstinputlisting{./examples/basic/ObjectPool.h}
\end{column}

\begin{column}{.45\textwidth}
\lstinputlisting{./examples/basic/ObjectPool.cpp}
\end{column}
\end{columns}
\end{frame}

\begin{frame}{Basic Implementation}
\lstinputlisting{./examples/basic/main.cpp}
\end{frame}

\begin{frame}{Basic Implementation}
	\begin{itemize}
		\setlength\itemsep{1em}
		\item ObjectPool class is monostate (or singleton)
		\item Put a flag in SharedObject to know used or not
		\item In AcquireObject(): check if an existing object is available in the pool and used or not?
		\item In ReleaseObject(): compare object with the pool and change the flag
	\end{itemize}
\end{frame}

\begin{frame}{Game Example}

\end{frame}

\begin{frame}{Game Example with ObjectPool}

\end{frame}

\begin{frame}{Multiple Objects in ObjectPool}

\end{frame}

\begin{frame}{ObjectPool with Factory Method}

\end{frame}

\begin{frame}{Generic ObjectPool}

\end{frame}

\begin{frame}{Pros and Cons}	
\begin{columns}[T]
\begin{column}{.45\textwidth}
	Pros
	\begin{itemize}
		\setlength\itemsep{1em}
		\item Reduce coupling with concrete classes
		\item Behave like operator new, but more flexible
		\item Caching existing instances improves performance of the app
		\item Reduce the overhead of heap allocation and deallocation
		\item Reduce heap fragmentation
	\end{itemize}
\end{column}
\begin{column}{.45\textwidth}
	Cons	
	\begin{itemize}
		\setlength\itemsep{1em}
		\item Memory may be wasted on unused pooled objects
		\item Pooled objects may remain in memory until the end of program
		\item Objects that are acquired from the pool must be reset prior to their use
		\item Clients have to ensure that an unused object is returned to the pool
		\item ObjectPool class may get tightly coupled with the classed of the pooled objects
	\end{itemize}
\end{column}
\end{columns}
\end{frame}

\begin{frame}{Where to use?}	
	\begin{itemize}
		\setlength\itemsep{1em}
		\item Frequently create and destroy objects
		\item Allocating heap objects is slow
		\item Objects are expensive to create
	\end{itemize}
\end{frame}

\end{document}

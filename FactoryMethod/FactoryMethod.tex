\documentclass[13pt]{beamer}
%
% Choose how your presentation looks.
%
% For more themes, color themes and font themes, see:
% http://deic.uab.es/~iblanes/beamer_gallery/index_by_theme.html
%
\mode<presentation>
{
  \usetheme{CambridgeUS}     % or try Darmstadt, Madrid, Warsaw, ...
  \usecolortheme{beaver} % or try albatross, beaver, crane, ...
  \usefonttheme{default}  % or try serif, structurebold, ...
  \setbeamertemplate{navigation symbols}{}
  \setbeamertemplate{caption}[numbered]
} 

\usepackage[english]{babel}
\usepackage[utf8x]{inputenc}
\usepackage{xcolor}
\usepackage{multicol}
\usepackage{tikz}
\usepackage{tikz-uml}
\tikzumlset{font=\footnotesize\ttfamily, class width=6ex}
\usepackage{hyperref}

\usepackage{listings}
\definecolor{codegreen}{rgb}{0,0.6,0}
\definecolor{codegray}{rgb}{0.5,0.5,0.5}
\definecolor{codepurple}{rgb}{0.58,0,0.82}
\definecolor{backcolour}{rgb}{0.95,0.95,0.92}

\lstdefinestyle{myCustomCppStyle}{
  language=C++,
  numbers=left,
  stepnumber=1,
  numbersep=9pt,
  tabsize=2,
  showspaces=false,
  showstringspaces=false
}

\lstset{basicstyle=\tiny,style=myCustomCppStyle}

\lstdefinestyle{mystyle}{
    backgroundcolor=\color{backcolour},   
    commentstyle=\color{codegreen},
    keywordstyle=\color{magenta},
    numberstyle=\tiny\color{codegray},
    stringstyle=\color{codepurple},
    basicstyle=\ttfamily\footnotesize,
    breakatwhitespace=false,         
    breaklines=true,                 
    captionpos=b,                    
    keepspaces=true,                 
    numbers=left,                    
    numbersep=5pt,                  
    showspaces=false,                
    showstringspaces=false,
    showtabs=false,                  
    tabsize=1
}

\lstset{style=mystyle}

\usepackage{graphicx}
\graphicspath{ {./images/} }

\usepackage{tikz}
\usetikzlibrary{decorations.text}
\usetikzlibrary{shapes.geometric, arrows, positioning, calc, matrix}

\tikzset{
  basic box/.style={
    shape=rectangle, rounded corners, align=center,
    draw=#1, fill=#1!25},
  header node/.style={
    Minimum Width=header nodes,
    font=\strut\Large\ttfamily,
    text depth=+0pt,
    fill=white, draw},
  header/.style={%
    inner ysep=+1.5em,
    append after command={
      \pgfextra{\let\TikZlastnode\tikzlastnode}
      node [header node] (header-\TikZlastnode) at (\TikZlastnode.north) {#1}
      node [span=(\TikZlastnode)(header-\TikZlastnode)] at (fit bounding box) (h-\TikZlastnode) {}
    }
  },
  hv/.style={to path={-|(\tikztotarget)\tikztonodes}},
  vh/.style={to path={|-(\tikztotarget)\tikztonodes}},
  fat blue line/.style={ultra thick, blue}
}

\definecolor{mygray}{RGB}{208,208,208}
\definecolor{mymagenta}{RGB}{226,0,116}
\newcommand*{\mytextstyle}{\sffamily\Large\bfseries\color{black!85}}
\newcommand{\arcarrow}[3]{%
   % inner radius, middle radius, outer radius, start angle,
   % end angle, tip protusion angle, options, text
   \pgfmathsetmacro{\rin}{1.7}
   \pgfmathsetmacro{\rmid}{2.2}
   \pgfmathsetmacro{\rout}{2.7}
   \pgfmathsetmacro{\astart}{#1}
   \pgfmathsetmacro{\aend}{#2}
   \pgfmathsetmacro{\atip}{5}
   \fill[mygray, very thick] (\astart+\atip:\rin)
                         arc (\astart+\atip:\aend:\rin)
      -- (\aend-\atip:\rmid)
      -- (\aend:\rout)   arc (\aend:\astart+\atip:\rout)
      -- (\astart:\rmid) -- cycle;
   \path[
      decoration = {
         text along path,
         text = {|\mytextstyle|#3},
         text align = {align = center},
         raise = -1.0ex
      },
      decorate
   ](\astart+\atip:\rmid) arc (\astart+\atip:\aend+\atip:\rmid);
}
\title[Design Pattern]{Creational Design Pattern}
\author{Hung Tran}
\institute{Fpt software}
\date{\today}

\begin{document}

\begin{frame}
  \titlepage
\end{frame}

% Uncomment these lines for an automatically generated outline.
\begin{frame}{Outline}
  \tableofcontents
\end{frame}

\section{Creational Pattern Overview}

\begin{frame}{Creational Pattern Overview}
	\begin{center}
	\textcolor{blue}{\textbf{Construction process of an object.}}
	\end{center}
	\begin{itemize}
		\setlength\itemsep{1em}
		\item \textbf{Singleton}: Ensure only one instance.
		\item \textbf{Factory Method}: Create instance without depending on its concrete type.
		\item \textbf{Object pool}: Reuse existing instances.
		\item \textbf{Abstract factory}: Create instances from a specific family.
		\item \textbf{Prototype}: Clone existing objects from a prototype.
		\item \textbf{Builder}: Construct a complex object step by step.
	\end{itemize}
\end{frame}

\section{Factory Method Pattern}

\begin{frame}{Logistic App}
\includegraphics[scale=0.5]{./images/truckLogistic.png}
\end{frame}

\begin{frame}{Logistic App}
	\begin{center}
	\begin{tikzpicture}
	\umlemptyclass[x=0,y=0]{Truck Class}
	\umlemptyclass[x=6,y=0]{LogisticApp Class}
	\umluniassoc[arg=bb, mult=1, pos=0.95, align=right, name=uniassoc]{LogisticApp Class}{Truck Class}
	\end{tikzpicture}	
	\end{center}
	\begin{itemize}
	\item Your app can only handle transportation by trucks.
	\item The bulk of your code lives inside the Truck class.
	\end{itemize}
	\begin{center}
	\textcolor{red}{You have more request from oversea regions, how to incorporate sea transportation with current app?}
	\end{center}
	\begin{itemize}
	\item At present, your code is tightly coupled with Truck class.
	\item Adding new \textcolor{red}{Ship class} resulting in change all code base.
	\item Logistic app depends on transportation object.
	\end{itemize}
\end{frame}

\begin{frame}{Using "new" operator}
\begin{columns}[T]
\begin{column}{.45\textwidth}
\lstset{basicstyle=\tiny,style=myCustomCppStyle}
\lstinputlisting{./examples/newOperator/newOperator.cpp}
\end{column}
\begin{column}{.48\textwidth}
	\begin{itemize}
		\setlength\itemsep{1em}
		\item Need name of class
		\item Tightly coupled with the name
		\item Add new class, modify the existing code
		\item Compiler does not know which instance created at compile time or an instance has to be created at runtime?
	\end{itemize}
\end{column}
\end{columns}
\end{frame}

\begin{frame}{The Intent of Factory Method Design Pattern}
	\begin{center}
	\textcolor{blue}{\textbf{Define an interface for creating an object, but let subclasses which class to instantitate. Factory method lets class defer instantiation to subclasses.}}
	\end{center}
\end{frame}

\begin{frame}{How to Implement of Factory Method Design Pattern?}
	\begin{itemize}
		\setlength\itemsep{1em}
		\item Different ways to implement
		\item An overridable method is provide that returns an instance of a class
		\item This method can be overridden to return instance of a subclass
		\item Behave likes \textbf{constructor}
		\item However, the constructor always returns the same instance
		\item The factory method can returns any sub-type 
		\item The factory method also called \textbf{virtual constructor}
		\item C++ language does not allow virtual constructor
	\end{itemize}
\end{frame}

\begin{frame}{Structure of Factory Method Design Pattern}
	\begin{center}
	\begin{tikzpicture}
	\umlemptyclass[type=abstract,x=0,y=0]{Product}
	\umlemptyclass[x=0,y=-3]{ConcreteProduct}
	\umlinherit{ConcreteProduct}{Product}
	
	\umlclass[type=abstract,x=4,y=0]{Creator}{}{\umlvirt{FactoryMethod()} \\ AnOperation()}
	\umlclass[x=4,y=-3]{ConcreteCreator}{}{FactoryMethod()}
	\umlinherit{ConcreteCreator}{Creator}
	\umlimport[geometry = -|-, weight = 0.3]{ConcreteCreator}{ConcreteProduct}
	\umlnote[x=8, y=-1, width = 4.2cm]{Creator}{product = FactoryMethod()}
	\umlnote[x=8, y=-4, width = 3.5cm]{ConcreteCreator}{return new concrete product}
	\end{tikzpicture}	
	\end{center}
\end{frame}

\begin{frame}{Modify existing code problem}
	\begin{center}
	\begin{tikzpicture}
	\umlemptyclass[type=abstract,x=0,y=0]{Product}
	\umlemptyclass[x=0,y=-3]{ConcreteProduct}
	\umlinherit{ConcreteProduct}{Product}
	\umlclass[type=abstract,x=6,y=0]{Creator}{}{AnOperation():Product*}
	\umluniassoc[pos=0.95, align=right, name=uniassoc]{Creator}{Product}
	\umluniassoc[pos=0.95, align=right, geometry = -|-, name=uniassoc]{Creator}{ConcreteProduct}
	\end{tikzpicture}	
	\end{center}
	\begin{itemize}
	\item Operation() method returns ConcreteProduct object
	\item Add more ConcreteProduct class, modify AnOperation() method
	\item End up with if else condition.
	\item Always return the same class (using new operator)
	\end{itemize}
\end{frame}

\iffalse
\begin{frame}{Modify existing code problem}
\begin{columns}[T]
\begin{column}{.45\textwidth}
\lstset{basicstyle=\tiny,style=myCustomCppStyle}
Product.h
\lstinputlisting{./examples/basicImplementation/naiveBasic/Product.h}
ConcreteProduct.h
\lstinputlisting{./examples/basicImplementation/naiveBasic/ConcreteProduct.h}
ConcreteProduct.cpp
\lstinputlisting{./examples/basicImplementation/naiveBasic/ConcreteProduct.cpp}
\end{column}

\begin{column}{.45\textwidth}
\lstset{basicstyle=\tiny,style=myCustomCppStyle}
Creator.h
\lstinputlisting{./examples/basicImplementation/naiveBasic/Creator.h}
Creator.cpp
\lstinputlisting{./examples/basicImplementation/naiveBasic/Creator.cpp}
main.cpp
\lstinputlisting{./examples/basicImplementation/naiveBasic/main.cpp}
\end{column}
\end{columns}
\end{frame}

\begin{frame}{What if we add one more ConcreteProduct class?}
\begin{columns}[T]
\begin{column}{.45\textwidth}
\lstset{basicstyle=\tiny,style=myCustomCppStyle}
ConcreteProduct1.h
\lstinputlisting{./examples/basicImplementation/naiveBasic/ConcreteProduct1.h}
ConcreteProduct1.cpp
\lstinputlisting{./examples/basicImplementation/naiveBasic/ConcreteProduct1.cpp}
\end{column}

\begin{column}{.45\textwidth}
\lstset{basicstyle=\tiny,style=myCustomCppStyle}
Creator.cpp
\lstinputlisting{./examples/basicImplementation/naiveBasic/Creator.cpp}
\textcolor{red}{\textbf{Factory Method Design Pattern comes in handy}}
\end{column}
\end{columns}
\end{frame}

\begin{frame}{Basic Implementation}
\begin{columns}[T]
\begin{column}{.45\textwidth}
\lstset{basicstyle=\tiny,style=myCustomCppStyle}
Product.h
\lstinputlisting{./examples/basicImplementation/factoryBasics/Product.h}
ConcreteProduct.h
\lstinputlisting{./examples/basicImplementation/factoryBasics/ConcreteProduct.h}
ConcreteProduct.cpp
\lstinputlisting{./examples/basicImplementation/factoryBasics/ConcreteProduct.cpp}
\end{column}

\begin{column}{.45\textwidth}
\lstset{basicstyle=\tiny,style=myCustomCppStyle}
ConcreteProduct1.h
\lstinputlisting{./examples/basicImplementation/factoryBasics/ConcreteProduct1.h}
ConcreteProduct1.cpp
\lstinputlisting{./examples/basicImplementation/factoryBasics/ConcreteProduct1.cpp}
\end{column}
\end{columns}
\end{frame}

\begin{frame}{Basic Implementation}
\begin{columns}[T]
\begin{column}{.45\textwidth}
\lstset{basicstyle=\tiny,style=myCustomCppStyle}
Creator.h
\lstinputlisting{./examples/basicImplementation/factoryBasics/Creator.h}
Creator.cpp
\lstinputlisting{./examples/basicImplementation/factoryBasics/Creator.cpp}
\end{column}

\begin{column}{.45\textwidth}
\lstset{basicstyle=\tiny,style=myCustomCppStyle}
ConcreteCreator.h
\lstinputlisting{./examples/basicImplementation/factoryBasics/ConcreteCreator.h}
ConcreteCreator.cpp
\lstinputlisting{./examples/basicImplementation/factoryBasics/ConcreteCreator.cpp}
\end{column}
\end{columns}
\end{frame}

\begin{frame}{Basic Implementation of Factory Method Pattern}
\begin{columns}[T]
\begin{column}{.45\textwidth}
\lstset{basicstyle=\tiny,style=myCustomCppStyle}
ConcreteCreator1.h
\lstinputlisting{./examples/basicImplementation/factoryBasics/ConcreteCreator1.h}
ConcreteCreator1.cpp
\lstinputlisting{./examples/basicImplementation/factoryBasics/ConcreteCreator1.cpp}
\end{column}

\begin{column}{.45\textwidth}
\lstset{basicstyle=\tiny,style=myCustomCppStyle}
main.cpp
\lstinputlisting{./examples/basicImplementation/factoryBasics/main.cpp}
\end{column}
\end{columns}
\end{frame}
\fi

\begin{frame}{Class Diagram Explaining}
\begin{center}
\begin{tikzpicture}
	\umlemptyclass[type=abstract,x=0,y=1]{Product}
	\umlemptyclass[x=-2,y=-2]{ConcreteProduct}
	\umlemptyclass[x=2,y=-2]{ConcreteProduct1}
	\umlinherit[geometry=-|]{ConcreteProduct}{Product}
	\umlinherit[geometry=-|]{ConcreteProduct1}{Product}
	
	\umlclass[type=abstract,x=5,y=1]{Creator}{m\_pProduct: Product*}{\umlvirt{Create()} \\ AnOperation()}
	\umlclass[x=3,y=-4]{ConcreteCreator}{}{Create()}
	\umlclass[x=7,y=-4]{ConcreteCreator1}{}{Create()}
	\umlinherit[geometry=-|]{ConcreteCreator}{Creator}
	\umlinherit[geometry=-|]{ConcreteCreator1}{Creator}
	
	\umluniassoc[geometry=|-, color=green]{Creator}{Product}
	
	\umlimport[geometry = -|, weight = 0.5, color=blue]{ConcreteCreator}{ConcreteProduct}
	\umlimport[geometry = |-, weight = 0.5, color=red]{ConcreteCreator1}{ConcreteProduct1}
\end{tikzpicture}	
\end{center}
\end{frame}

\begin{frame}{Create a ConcreteProduct object}
\begin{columns}[T]
\begin{column}{.4\textwidth}
\lstset{basicstyle=\tiny,style=myCustomCppStyle}
main.cpp
\lstinputlisting{./examples/basicImplementation/factoryBasics/main.cpp}
\end{column}
\begin{column}{.6\textwidth}
	\begin{itemize}
		\setlength\itemsep{1em}
		\item Create() Method (or factory method) is overridden in derived classes.
		\item The dependency on ConcreteProduct is no longer required
		\item AnOperation method calls Create() method.
		\item With the main function, which Create() method will invoke?
		\item No need to modify the AnOperation()
		\item Create() method behaves like virtual constructor
	\end{itemize}
\end{column}
\end{columns}
\end{frame}

\begin{frame}{Real World Example: Application Framework?}
	\textcolor{blue}{We want to create an framework!}	
	\begin{itemize}
		\setlength\itemsep{1em}
		\item The programmer can create an Application based on App framework.
		\item Framework provides managing different kinds of documents.
		\item Framework provides infrastructure of application.
		\item By using app framework, programmer can create an application which can display different kinds of data (\textcolor{red}{Text or Graphics}).
		\item To manage the data, the framework provides support through which is easy to maintain data.
		\item We add \textcolor{red}{Application class} for managing data and display them.
		\item We need add an \textcolor{red}{Document class} that Application manages
	\end{itemize}
\end{frame}

\begin{frame}{Real World Example: Application Framework?}
\begin{itemize}
	\setlength\itemsep{1em}
	\item Application class uses the features of Document class (display them).
	\item However, Application does not know what kind of data is.
	\item Document class need to be abstract in our framework.
	\item \textbf{Document class has methods Read(), Write() which are overridden in derived classes.}
	\item \textbf{Application class uses the features of Document class.}
\end{itemize}
\begin{center}
\begin{tikzpicture}
	\umlclass[type=abstract,x=0,y=0]{Document}{}{\umlvirt{Write()} \\ \umlvirt{Read()}}
	\umlclass[x=5,y=0]{Application}{}{}
	\umluniassoc[geometry=|-]{Application}{Document}
\end{tikzpicture}	
\end{center}
\end{frame}

\begin{frame}{Using Framework for managing data}
\begin{itemize}
	\setlength\itemsep{1em}
	\item Programmer want to use the framework for managing the Text data.
	\item Create a class Text which is derived class of Document class.
	\item Application can use Text through Document class.
\end{itemize}
\begin{center}
\begin{tikzpicture}
	\umlclass[type=abstract,x=0,y=0]{Document}{}{\umlvirt{Write()} \\ \umlvirt{Read()}}
	\umlclass[x=0,y=-2.5]{Text}{}{Write() \\ Read()}
	\umlinherit[geometry=--]{Text}{Document}
	
	\umlclass[x=6,y=0]{Application}{}{New() \\ Open() \\ Save()}
	\umluniassoc[geometry=|-]{Application}{Document}
\end{tikzpicture}	
\end{center}
\end{frame}

\begin{frame}{Adding SpreadSheet class}
\begin{itemize}
	\setlength\itemsep{1em}
	\item New(), Open(), Save() are tight coupled with Text class.
	\item Adding new Document class, change the Application class.
	\item The framework does not allow modifying.
\end{itemize}
\begin{center}
\begin{tikzpicture}
	\umlclass[type=abstract,x=0,y=0]{Document}{}{\umlvirt{Write()} \\ \umlvirt{Read()}}
	\umlclass[x=-2,y=-2.5]{Text}{}{Write() \\ Read()}
	\umlclass[x=2,y=-2.5]{SpreadSheet}{}{Write() \\ Read()}
	\umlinherit[geometry=-|]{Text}{Document}
	\umlinherit[geometry=-|]{SpreadSheet}{Document}
	
	\umlclass[x=6,y=0]{Application}{}{New() \\ Open() \\ Save()}
	\umluniassoc[geometry=|-]{Application}{Document}
\end{tikzpicture}	
\end{center}
\end{frame}

\begin{frame}{App framework with factory method}
\begin{center}
\begin{tikzpicture}
	\umlclass[type=abstract,x=0,y=1]{Document}{}{\umlvirt{Read()} \\ \umlvirt{Write()}}
	\umlemptyclass[x=-2,y=-2]{Text}
	\umlemptyclass[x=2,y=-2]{SpreadSheet}
	\umlinherit[geometry=-|]{Text}{Document}
	\umlinherit[geometry=-|]{SpreadSheet}{Document}
	
	\umlclass[x=5,y=1]{Application}{m\_pDocument: Document*}{\umlvirt{Create()}: Document* \\ New() \\ Open() \\ Save()}
	\umlclass[x=2,y=-4]{TextApp}{}{Create(): Document*}
	\umlclass[x=7.3,y=-4]{SpreadSheetApp}{}{Create(): Document*}
	\umlinherit[geometry=-|]{TextApp}{Application}
	\umlinherit[geometry=-|]{SpreadSheetApp}{Application}
	
	\umluniassoc[geometry=|-, color=green]{Application}{Document}
	
	\umlimport[geometry = -|, weight = 0.5, color=blue]{TextApp}{Text}
	\umlimport[geometry = |-, weight = 0.5, color=red]{SpreadSheetApp}{SpreadSheet}
\end{tikzpicture}	
\end{center}
\end{frame}

\begin{frame}{The Intent of Factory Method Design Pattern}
	\begin{center}
	\textcolor{blue}{\textbf{Define an interface for creating an object, but let subclasses which class to instantitate. Factory method lets class defer instantiation to subclasses.}}
	\end{center}
	\begin{itemize}
	\item Application class is an interface.
	\item Create() method is factory method.
	\item The subclasses (TextApp and SpreadSheetApp) instantiate objects.
	\end{itemize}
\end{frame}

\begin{frame}{The Intent of Factory Method Design Pattern}
\begin{columns}[T]
\begin{column}{.4\textwidth}
\lstset{basicstyle=\tiny,style=myCustomCppStyle}
\lstinputlisting{./examples/appFramework/appFrame//TextApplication.cpp}
\end{column}
\begin{column}{.6\textwidth}
	\begin{itemize}
		\setlength\itemsep{1em}
		\item \textcolor{red}{How to free the instance?}
		\item Using smart pointers
	\end{itemize}
\end{column}
\end{columns}
\end{frame}

\begin{frame}{Using smart pointer}
\begin{center}
\begin{tikzpicture}
	\umlclass[type=abstract,x=0,y=1]{Document}{}{\umlvirt{Read()} \\ \umlvirt{Write()}}
	\umlemptyclass[x=-2,y=-2]{Text}
	\umlemptyclass[x=2,y=-2]{SpreadSheet}
	\umlinherit[geometry=-|]{Text}{Document}
	\umlinherit[geometry=-|]{SpreadSheet}{Document}
	
	\umlclass[x=5,y=1]{Application}{m\_pDocument: unique\_ptr<Document>}{\umlvirt{Create()}: unique\_ptr<Document> \\ New() \\ Open() \\ Save()}
	\umlclass[x=2,y=-4]{TextApp}{}{Create(): unique}
	\umlclass[x=7.3,y=-4]{SpreadSheetApp}{}{Create(): unique}
	\umlinherit[geometry=-|]{TextApp}{Application}
	\umlinherit[geometry=-|]{SpreadSheetApp}{Application}
	
	\umluniassoc[geometry=|-, color=green]{Application}{Document}
	
	\umlimport[geometry = -|, weight = 0.5, color=blue]{TextApp}{Text}
	\umlimport[geometry = |-, weight = 0.5, color=red]{SpreadSheetApp}{SpreadSheet}
\end{tikzpicture}
\end{center}
\end{frame}

\begin{frame}{Add new Document, add more Application?}
	\textcolor{blue}{How to create multiple instances without creating corresponding application class?}	
	\begin{center}
	\textcolor{red}{\textbf{Using Parameterized Factory}}
	\end{center}
\end{frame}

\begin{frame}{What if you want to read different kinds of document?}
\begin{center}
\includegraphics[scale=0.25]{./images/visualDialog.png}
\end{center}

\begin{itemize}
	\setlength\itemsep{1em}
	\item Like dialog shows options which clients can chose.
	\item You need to know how many applications before implementations.
	\item Input Type through user interface.
\end{itemize}
\end{frame}

\begin{frame}{Classes structure: Parameterized Factory}
\begin{center}
\begin{tikzpicture}
	\umlclass[type=abstract,x=-1.5,y=2]{Document}{}{\umlvirt{Read()} \\ \umlvirt{Write()}}
	\umlemptyclass[x=-3,y=-2]{Text}
	\umlemptyclass[x=1,y=-2]{SpreadSheet}
	\umlinherit[geometry=-|]{Text}{Document}
	\umlinherit[geometry=-|]{SpreadSheet}{Document}
	
	\umlclass[x=3,y=0]{DocFactory}{}{Create(Type): Document*}
	\umlclass[x=7,y=-2]{Application}{DocFactory}{New() \\ Open() \\ Save()}
	\umluniassoc[geometry=|-, color=green]{Application}{DocFactory}
	\umlimport[geometry = -|, weight = 0.5, color=blue]{DocFactory}{Text}
	\umlimport[geometry = |-, weight = 0.5, color=red]{DocFactory}{SpreadSheet}
\end{tikzpicture}
\end{center}
\end{frame}

\iffalse
\begin{frame}{App framework: parameterized Factory}
\begin{columns}[T]
\begin{column}{.45\textwidth}
\lstset{basicstyle=\tiny,style=myCustomCppStyle}
Document.h
\lstinputlisting{./examples/parameterizedFactory/paraFact/Document.h}
TextDocument.h
\lstinputlisting{./examples/parameterizedFactory/paraFact/TextDocument.h}
\end{column}

\begin{column}{.45\textwidth}
\lstset{basicstyle=\tiny,style=myCustomCppStyle}
TextDocument.cpp
\lstinputlisting{./examples/parameterizedFactory/paraFact/TextDocument.cpp}
\end{column}
\end{columns}
\end{frame}

\begin{frame}{App framework: parameterized Factory}
\begin{columns}[T]
\begin{column}{.45\textwidth}
\lstset{basicstyle=\tiny,style=myCustomCppStyle}
SpreadSheetDocument.h
\lstinputlisting{./examples/parameterizedFactory/paraFact/SpreadSheetDocument.h}
SpreadSheetDocument.cpp
\lstinputlisting{./examples/parameterizedFactory/paraFact/SpreadSheetDocument.cpp}
\end{column}

\begin{column}{.45\textwidth}
\lstset{basicstyle=\tiny,style=myCustomCppStyle}
\end{column}
\end{columns}
\end{frame}

\begin{frame}{App framework: parameterized Factory}
\begin{columns}[T]
\begin{column}{.45\textwidth}
\lstset{basicstyle=\tiny,style=myCustomCppStyle}
Application.h
\lstinputlisting{./examples/parameterizedFactory/paraFact/Application.h}
\end{column}

\begin{column}{.45\textwidth}
\lstset{basicstyle=\tiny,style=myCustomCppStyle}
Application.cpp
\lstinputlisting{./examples/parameterizedFactory/paraFact/Application.cpp}
\end{column}
\end{columns}
\end{frame}
\fi

\begin{frame}{Pros and Cons}
\begin{columns}[T]
\begin{column}{.45\textwidth}
	\begin{center}
	\textcolor{blue}{\textbf{Pros}}
	\end{center}
	\begin{itemize}
		\item Instances can be created at runtime
		\item Promote loose coupling
		\item Construction becomes simple due to abstraction
		\item Construction becomes encapsulated
		\item May not return new instance every time (return a cache instance), useful for object pool
	\end{itemize}
\end{column}
\begin{column}{.45\textwidth}
	\begin{center}
	\textcolor{blue}{\textbf{Cons}}
	\end{center}
		\begin{itemize}
		\setlength\itemsep{1em}
		\item Every new product class may require a corresponding factory class.
	\end{itemize}
\end{column}
\end{columns}
\end{frame}

\begin{frame}{Where to use?}	
	\begin{itemize}
		\setlength\itemsep{1em}
		\item A class does not know which instance it needs at runtime.
		\item A class does not want to depend on concrete classes that it uses.
		\item You want to encapsulate the creation process.
	\end{itemize}
\end{frame}

\end{document}

%******************************section 1.2*************************************************
%
%
\section{Sequence Containers} %section 1.2
\subsection{Sequence Containers} %subsection 1.2.1
These containers maintain a certain order to the elements inside them. This order can be completely 
controlled by a programmer, and he or she can choose the exact position at which a particular element 
will be located/placed in a sequence container. The containers themselves have no influence on the 
sequence of elements.

In this category, the STL offers three solutions:
\begin{itemize}
    \item vector
    \item deque
    \item list
\end{itemize}
Before we describe the containers in detail, let’s take a look at the table, which shows the various 
methods provided by each of them. As you can see, most of the methods are common among all three containers. 
But don’t worry, each of the methods will be described appropriately, and with an accompanying example.

\subsection{\texttt{Vectors}} %subsection 1.2.2
\begin{classinfo}
  {<vector>}{template<class T, class Allocator = std::allocator<T>> class vector;}
\end{classinfo}
The vector is usually the first container to learn when somebody starts his or her adventure with the STL. 
A vector is basically a dynamic array. It always occupies a continuous memory block. The example at 
the start of this tutorial shows how a dynamic array works. The biggest advantage of the vector is 
the ability to access its elements randomly (by means of the index) in constant time.

The \texttt{allocator} is responsible for providing a \textbf{memory model} for the container elements. 
Usually the default one is used (\inlinecode{C++}{std::allocator<T>}), but it’s also possible to supply 
a different one. Technically, you’re allowed to use a different \texttt{allocator} for each type of container.

\textbf{The definition of a vector shows that a vector is a template class} and \textbf{it’s necessary 
to specify the type of its elements during instantiation}.
For example:
\begin{lstlisting}[language=C++]
class A;
vector<int> v1;
vector<float> v2;
vector<A> v3;
\end{lstlisting}

\subsection{Constructing \texttt{vectors}} %subsection 1.2.3
Before we can start using the \texttt{vector}, we need to create it. We already know that we need 
to choose the type of elements of the \texttt{vector}, but this isn’t the only choice we have to make. 
The \texttt{vector} class possesses a few constructor methods, which can be used as needed:
\begin{lstlisting}[language=C++]
explicit vector ( const Allocator& = Allocator() );
explicit vector ( size_type n, const T& value= T(), \
  const Allocator& = Allocator() );
template <class InputIterator>
  vector ( InputIterator first, InputIterator last, \
    const Allocator& = Allocator() );
  vector ( const vector<T,Allocator>& x );
\end{lstlisting}

\textbf{Default constructor}

The first constructor is the default constructor. We’ve seen its usage before. The \texttt{allocator} 
parameter is optional in all the constructors.

\textbf{Initializing the constructor}

The second constructor creates a \texttt{allocator} containing n objects, all of them having the same 
value – value. It can be used in situations where someone needs a vector of a predefined size and wants 
to initialize it. Look at the example.

\textcolor{green}{File name: 1.2.4.cpp}
\lstinputlisting[language=C++]{Chapter1/codes/1.2.4.cpp}

\subsection{Constructing vectors Iterator constructors} %subsection 1.2.4
The next constructor uses \texttt{iterators} to initialize itself. It will create a 
\texttt{vector} with a copy of the values from first (inclusive) to last (exclusive). 
In the most typical case, this constructor creates a new \texttt{vector} using the 
elements from an already existing collection.

But due to the fact that \texttt{iterators} are defined as a set of operations instead 
of a certain type, it is also possible to use normal pointers. This remark leads us to 
the conclusion that you can use a normal C++ array to initialize a collection. 
And in fact, you can.

\subsection{Iterator constructors – initialization} %subsection 1.2.5
As you can see, you can initialize a vector using a range of elements from another vector 
(or a collection in general) or an array. We need to remember again that the first element 
will be included, and the last element will not be in that range.

\textcolor{green}{File name: 1.2.5.cpp}
\lstinputlisting[language=C++]{Chapter1/codes/1.2.5.cpp}

\subsection{Copy constructors} %subsection 1.2.6
The copy constructor is a very important concept in the C++ world, so it’s not surprising 
that a vector also possesses one. It works in an obvious way, creating a new object from the existing one.

The copy is exact; therefore, every element from the source is copied. It’s a very important fact, so 
remember this for now. On this page, there is an example of a copy constructor. Nothing fancy, but 
it shows the idea how the copy constructor is supposed to work.

On the next page, we’ll show you a more advanced example.

\textcolor{green}{File name: 1.2.6.cpp}
\lstinputlisting[language=C++]{Chapter1/codes/1.2.6.cpp}

\subsection{Dealing with objects} %subsection 1.2.7
In all the examples presented so far, the \texttt{vectors} have been defined as of type \texttt{int}.  
Now we’ll have to take a look at what happens when the type is not a C++ built-in one. There are many 
things that have to be taken into consideration.

Let’s go through all the constructors again, but this time, let’s create a \texttt{vector} of the objects. 
We need to pay attention to the behavior of the constructors (normal, default and copy as well) and 
assignment operators. Also, some aspects of type conversion will come into the picture.

Take a look at the example. The topic is: \textbf{\texttt{implicit} type conversion}.
What do you think of this example? Will it compile or not? The answer is yes, it will.
There is this constructor inside the class, which takes the int parameter. 

The \inlinecode{C++}{push_back()} method expects an object of type \texttt{A}. This constructor will 
be used to create an object of type \texttt{A implicitly}. This can be an advantage, but it might also 
pose a danger if that constructor does something unwanted.

For example, an object might be created \texttt{implicitly} and passed to a method when we do not want 
it to happen in such a way. This would not be allowed if the constructor was marked \texttt{explicit}. 
Let’s see this on the next page.

\textcolor{green}{File name: 1.2.7.cpp}
\lstinputlisting[language=C++]{Chapter1/codes/1.2.7.cpp}

\subsection{Constructors and objects} %subsection 1.2.8
Let’s move on further. The purpose of this example is to show you explicitly when a particular 
type of constructor is being called. It’s not always obvious, therefore you might forget to 
create a constructor.

The numbers in brackets \texttt{()} indicate a particular line of code and the corresponding 
lines of output. Let’s try to analyze the code and its behavior.

The first line creates a \texttt{vector} of size one. We’ve used the initializing constructor. 
Its first argument is a number of elements, and the second one is a possible value to initialize 
the elements. If the value is not submitted, the default value is used. It that case, an object 
of class \texttt{A} is created using a default constructor. The object is then passed to the 
constructor to initialize one element of the collection.

In the second line, an object is created using the type conversion mechanism we described earlier. 
The one-parameter constructor is invoked. The next step happens inside the \texttt{vector v1}. 
First it reallocates itself to make room for another object. Then two objects – the new one and 
the old one – are copied into the \texttt{vector} storage area.

The third line is just an assignment. But again, the first step is to create an object, then 
to assign it to its proper place.

This example clearly shows how important it is to have proper constructors, the copy constructor 
and the assignment operator in a class which is meant to be put into STL containers. You won’t 
always need to create all these methods explicitly. On many occasions, it’ll be sufficient to 
rely on their default implementations.

\textcolor{green}{File name: 1.2.8.cpp}
\lstinputlisting[language=C++]{Chapter1/codes/1.2.8.cpp}

\subsection{Creating a copy of a vector} %subsection 1.2.9
The final example will show what is happening while copying the entire \texttt{vector}. We’re using 
the same class \texttt{A} as in the previous example.

Let’s carry out a short analysis of what’s going on in this code. Again, the lines of code and 
the corresponding output are marked.

Part one is easy: the \texttt{v1 vector} is created and one element is added to it.

Part two: the copy constructor is invoked. Inside the source \texttt{vector} there is only one element, 
so the class \texttt{A} copy constructor is called only once.

Part three: this one is a little bit surprising. The output clearly states that the copy constructor 
has been called, but we were expecting the assignment operator instead, were we not? The target vector 
is empty. It looks as though in such a case, the copy constructor is called instead of the assignment 
operator. This situation is worth remembering.

Part four – the last one: this is pretty straightforward. A \texttt{vector} containing two objects of 
class A is created, so there are two invocations of the default constructor and the copy constructor. 
The next step is an assignment operation, and since the source vector has only one element, only one 
invocation of the assignment operator is performed.

\textcolor{green}{File name: 1.2.9.cpp}
\lstinputlisting[language=C++]{Chapter1/codes/1.2.9.cpp}

\subsection{Destructors} %subsection 1.2.10
There’s not much to say about this method. It does the housework. It destroys all the objects stored 
inside a \texttt{vector} by calling their destructors, and then deallocates all storage capacity 
reserved by the vector.

But it’s important to remember that the destructor of elements will be called if and only if these 
elements are static (not pointers). In the opposite case, the user is responsible for this activity.

Take a look at the example on the slide. A lot has happened in this code. We should focus on the output 
after the Third ready! mark. We can see three invocations of the destructor. All of them have been caused 
by the destructor of the \texttt{vector}.

\textcolor{green}{File name: 1.2.10.cpp}
\lstinputlisting[language=C++]{Chapter1/codes/1.2.10.cpp}

\subsection{Destroying dynamically allocated objects} %subsection 1.2.11
Now, for comparison, let’s see what happens when dynamically allocated objects are in the collection.
As you can see, there are no destructors calls whatsoever. Dynamically allocated objects stored inside 
a collection will not be automatically destroyed when this collection is being destroyed. This can lead 
to a memory leak, which is already bad enough. But there are scenarios where unwanted objects can lead 
to more severe consequences.

Although this example involves the vector only, it works in exactly the same manner for all the other collections.

\textcolor{green}{File name: 1.2.11.cpp}
\lstinputlisting[language=C++]{Chapter1/codes/1.2.11.cpp}

\subsection{\texttt{deque} class} %subsection 1.2.12

\begin{classinfo}
  {<deque>}{template<class T, class Allocator = std::allocator<T>> class deque;}
\end{classinfo}
The name deque stands for double-ended queue. In many ways, this container is very similar to a vector.
\textbf{The most important difference is that it does not have to occupy contiguous memory space}. 
If a \texttt{vector} is basically an array inside, \textbf{\texttt{deque} is a double-linked list of arrays}. 
This allows for fast insertion at both ends of the container. There’s no need for reallocation, because 
a new array can always be added on either side. \\
\texttt{deque}, similarly to a vector, allows for random access using \texttt{operator []}. 
The \texttt{deque} template definition has exactly the same meaning as the \texttt{vector} does. 
The \texttt{Allocator} is responsible for the memory model.

\subsection{\texttt{deque} constructors – size constructors} %subsection 1.2.13
\texttt{deque} has four constructors, virtually identical to those of the \texttt{vector}:
\begin{lstlisting}[language=C++]
explicit deque ( const Allocator& = Allocator() );
explicit deque ( size_type n, const T& value= T(), \
  const Allocator& = Allocator() );
emplate <class InputIterator> 
deque ( InputIterator first, InputIterator last, \
const Allocator& = Allocator() );
deque ( const deque<T,Allocator>& x );
\end{lstlisting}

The mode of operation of each of these constructors is identical to their \texttt{vector} counterparts. 
Again we have: the default constructor, the initializing constructor, the iterator constructor and 
the copy constructor.

On this and the following slides you will see examples of deque’s usage of constructors.

\textcolor{green}{File name: 1.2.13.cpp}
\begin{lstlisting}[language=C++]
#include <deque>
#include <iostream>

using namespace std;

int main()
{
    deque<int> d1(10, 0);
    cout<<"Size: "<<d1.size()<<endl;
    for(unsigned i = 0; i < d1.size(); ++i)
    {
        cout<< d1[i]<<" ";
    }
    cout<<endl;
    return 0;
}
\end{lstlisting}

\subsection{\texttt{deque} – iterator constructors} %subsection 1.2.14
Iterator constructor example:

\textcolor{green}{File name: 1.2.14.cpp}
\lstinputlisting[language=C++]{Chapter1/codes/1.2.14.cpp}

\subsection{\texttt{deque} – copy constructors} %subsection 1.2.15
An example of copy constructor usage for the \texttt{deque} class.
Elements which might be stored inside a \texttt{deque} require the same consideration as 
those stored inside a \texttt{vector}. They need: a proper constructor, a copy constructor and 
an assignment operator, to work as expected.

\textcolor{green}{File name: 1.2.15.cpp}
\lstinputlisting[language=C++]{Chapter1/codes/1.2.15.cpp}

\subsection{\texttt{list}} %subsection 1.2.16
\begin{classinfo}
  {<list>}{template<class T, class Allocator = std::allocator<T>> class list;}
\end{classinfo}
\textbf{The \texttt{list} container is an implementation of the double-linked list principle}. 
Each element has pointers leading to the next and the previous ones in a \texttt{list} sequence. 
The storage is \textbf{not contiguous}; it’s likely that each element will be placed in a completely 
different memory area.

The \texttt{list} container allows for fast insertion and deletion anywhere inside the range of its elements.
Generally, all operations related to the change in the sequence of elements are less costly than 
in \texttt{vector} and \texttt{deque}. \textbf{The price to pay for this advantage is the lack of the 
random access mechanism – operator[]}.
But of course, it’s still possible to iterate through the collection using a general STL mechanism – 
the \texttt{iterator}.
Another drawback is the additional data required to keep linking information for each of the elements. 
\textbf{This results in additional memory consumption}. In specific cases this might be a significant problem.

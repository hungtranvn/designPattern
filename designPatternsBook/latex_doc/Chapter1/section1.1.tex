%*******************************************************************************
%*********************************** First Chapter *****************************
%*******************************************************************************

\chapter{STL sequential containers}  %Title of the First Chapter

\ifpdf
    \graphicspath{{Chapter1/Figs/Raster/}{Chapter1/Figs/PDF/}{Chapter1/Figs/}}
\else
    \graphicspath{{Chapter1/Figs/Vector/}{Chapter1/Figs/}}
\fi


%********************************** %First Section  **************************************

\section{Quick introduction to Standard Template Library.}
\subsection{Basic information about STL} %subsection 1.1.1

Note: This course is in BETA version. Even though we do our best effort to ensure 
the courseware is of very good quality, some occasional errors, spelling mistakes, 
and formatting issues might occur both in the learning resources and assessments. 
We apologize for all of them. The course will be fully updated in the first 
quarter of 2018, most probably in February. Thank you for your patience and 
being understanding.

\textbf{STL} stands for \textbf{Standard Template Library}. STL is an important 
part of the C++ development environment, and if you intend to be a C++ programmer, 
you should certainly learn how to use it.

What is STL good for? Generally speaking, it provides ready-to-use solutions for 
many programming problems. The word “standard” is also meaningful. It means that 
every C++ compiler is accompanied by \textbf{this software package}. So, as long 
as you use STL, your program portability is high (portability of C++ programs is 
a huge but unrelated topic). What problems can we solve with the help of the STL? 
Many, but not all – otherwise \textbf{we wouldn’t need Boost libraries (free and 
peer-reviewed portable C++ source libraries)} and all of the language improvements.

Basically, STL can be divided into two parts:
\begin{itemize}
    \item \textbf{Containers}, and 
    \item \textbf{Algorithms}
\end{itemize}

There are more features in the library, and we’ll explain them later in the course – 
don’t worry. We just think that \textbf{containers and algorithms are most important}, 
and that’s why we’re going to focus on them first and most extensively.

\subsection{Quick introduction to containers} %subsection 1.1.2

\textbf{Containers}, or \textbf{collections}, as they are sometimes called, are meant 
to contain something inside. This something is various kinds of data. Different types 
of containers store data in different manners.
This is good, because it allows programmers to choose the best collection for the task 
they face. We’ll talk more about this later. The \textbf{simplest container} you can 
think of is an \textbf{array}, for example:

\inlinecode{C++}{int a[10];}

The array above is of type \texttt{int}. \textbf{An array is a container which stores 
elements of the same type}, and usually \textbf{those elements occupy a continuous area of memory}.
Each element is placed one after the other, starting from a certain location in the memory. 
You can access a particular element of an array by its index using the \texttt{[] operator} – 
the square brackets operator.

For example:
\begin{lstlisting}[language=C++]
a[3] = 3;
cout<<a[3]<<endl;
\end{lstlisting}

An array is a very basic container. It has a few \textbf{limitations}:
\begin{itemize}
    \item the size of an array cannot be changed once instantiated;
    \item an array does not know its size – this information has to be stored in another variable;
    \item as a result of the previous point, an array also cannot check if it is properly accessed 
      – the index used in the operator cannot be checked;
    \item an array (one dimension) is organized as one big memory block.
\end{itemize}
The last point is not always a limitation; it all depends on the particular array usage scenario.

\subsection{Improving arrays} %subsection 1.1.3

Let us think for a while about how to improve our array and remove some of \textbf{its limitations}.
What could you do?

Let’s design a \texttt{class} named \texttt{Array}, which will be our new better array. 
This \texttt{class} will remove the first two limitations of the original array: 
\textbf{the inability to change the size}, and \textbf{the lack of knowledge of its 
size (may we can know the size by \texttt{sizeoff()})}. The class in its basic form 
might look like the one on the slide.

As you can see, the \texttt{Array} class is built around a regular C/C++ dynamically 
allocated array. But there’s more. We’ve used class abstraction to improve the ordinary array.
On the next slide, the newly created class will be put to use.

\textcolor{green}{File name: 1.1.3.h}
\lstinputlisting[language=C++]{Chapter1/codes/1.1.3.h}

\textcolor{green}{File name: 1.1.3.cpp}
\lstinputlisting[language=C++]{Chapter1/codes/1.1.3.cpp}

\subsection{Using our improved array} %subsection 1.1.4
The first step is to declare an object of type \texttt{Array}.
Its initial size is set to 10. The constructor allocates an array of ten elements 
inside object \texttt{A}.

The second interesting thing is that we use this object in the same way as a 
regular array \texttt{operator[]} is called for access. Such behavior is possible 
because the class implements \texttt{operator[]}.

The loops illustrated in the code are using the \texttt{getSize()} method to obtain 
the actual size of the Array object. This is another improvement over a standard C++ array.

\textcolor{green}{File name: use1.1.3.cpp}
\lstinputlisting[language=C++]{Chapter1/codes/use1.1.3.cpp}

Let’s go further. You’ve probably noticed two methods which have not been discussed yet: 
\texttt{add} and \texttt{delItem()}. Now we’re going to put them to work. 

Take a look at the example on the slide. This is the \texttt{main()} function, 
which we’ve seen before, with additional code. The code takes advantage of the methods 
\texttt{add} and \texttt{delItem}. These two methods allow our new array to grow and 
shrink as needed. Another array limitation seems to have been overcome. The last important 
modification is related to access control, which is provided inside \texttt{operator[]}. 
In case a program tries to access an element by an index which is out of scope, an exception is thrown.

\subsection{Better array – STL way} %subsection 1.1.5

Of course, the \texttt{Array} class is far from perfect. There’s a lot of other functionality 
that could be implemented here. First of all, the class could be created as a template class. 
It would allow other types than just integer. Secondly, some performance issues are present in 
the previous code. But this is just a brief introduction to the world of containers. 
Now let’s take a look at how STL fits into this picture. In order to do that, we’ll rewrite 
the \texttt{main()} function using a very basic STL container – a vector. As you can see, 
the \texttt{vector} class allows for the same behavior as our handmade \texttt{Array} class. 
Moreover, the \texttt{vector} is a template class and can work with basically 
\textbf{any type of element}. That is a \textbf{huge advantage}.

In this example, we’re using two very common \texttt{vector} methods: 
\inlinecode{C++}{push_back()} – which adds a new element to the end of a vector, 
and \inlinecode{C++}{pop_back()} – this one removes the last element from a vector.
As the \texttt{Array} class, the vector also has \texttt{operator[]} overloaded. 
And finally, the method \texttt{size()}, whose task is rather obvious. This is a vector, 
in a nutshell, ladies and gentlemen. Let’s see what else can be found inside the 
\textbf{Standard Template Library}. With the introduction of C++11, we can use a very 
compact way to initialize vector elements – the initializer list. We’ll present it here 
shortly; it’s just a comma-separated list enclosed in braces: \texttt{vector<int> v1 = {5, 6, 9};}

\textcolor{green}{File name: 1.1.5.cpp}
\lstinputlisting[language=C++]{Chapter1/codes/1.1.5.cpp}

\subsection{Standard Template Library} %subsection 1.1.6

The STL can be divided into the following parts:
\begin{itemize}
    \item containers
    \item algorithms
    \item input/output
    \item strings
    \item numeric library
    \item iterators
    \item utilities
    \item localizations
    \item regular expressions
    \item atomic operations
    \item thread support
    \item concepts
\end{itemize}

This division is not meant to be definitive. Different books or websites might provide different 
views of the STL. Let us take a closer look at each of the first seven categories.

\subsection{Containers} %subsection 1.1.7
This is one of two major parts of the STL. It is made up of various containers which help the 
programmer to implement the best solution to the task at hand.

The \textbf{Standard Template Library} provides the programmer with a lot of different containers, 
split into the following subcategories:
\begin{itemize}
    \item sequential containers:
    \begin{itemize}
        \item vector
        \item list
        \item deque
    \end{itemize}    
    \item associative containers:
    \begin{itemize}
        \item set
        \item multiset
        \item map
        \item multimap
    \end{itemize}
    \item container adaptors:
    \begin{itemize}
        \item stack
        \item queue
        \item priority queue
    \end{itemize}
\end{itemize}

If you’re familiar with abstract data structures, some of these names will be quite familiar – 
list, set, map, stack, etc. Containers are based on some underlying data abstraction, which 
is usually a well-known computer science concept. All this will be explained later.

\subsection{Algorithms} %subsection 1.1.8

The second biggest STL branch is \textbf{algorithms} – a means to transform data – 
usually stored inside either a particular or a general container. \textbf{These algorithms 
are expressed as functions}. There’s a huge number of them, and if you’re thinking about 
implementing an algorithm, it would be prudent to first check if it’s not already present 
in the STL. Once again, we need to emphasize \textbf{the importance of using a well-defined library}, 
instead of creating your own versions. \textbf{Containers are structures only. Without algorithms, 
they’re not of very much use}. Different sources divide STL algorithms into different categories. 
Below you can find a typical list:
\begin{itemize}
    \item non-modifying sequence operations
    \item modifying sequence operations
    \item sorting
    \item set operations
    \item binary search
    \item heap operations
    \item min/max operations
\end{itemize}
The list is quite impressive; it’s highly probable that you’ll find what you are looking for here.

\subsection{Input and output library} %subsection 1.1.9
This part of the STL is responsible for all kinds of input and output operations: 
\textbf{console and files}. You certainly know the \texttt{cout} object and the 
\texttt{iostream} include. This is exactly the STL i/o division.

\subsection{String library} %subsection 1.1.10
The string library is C++'s response to the char * problem. The class string is a solution 
to many problems related to the processing of character strings. The class is not perfect, 
especially if you’re familiar with its Java counterpart. There are also some performance 
considerations, but still, this class solves a lot of common string problems. Also, it’s worth 
mentioning that the string class is a container itself, and can be treated like one.

\subsection{Numeric library} %subsection 1.1.11
A few additional classes are strongly connected to the math world. There is \textbf{the complex class}, 
which is exactly what you would expect, and the valarray type. \textbf{Valarray} is a special kind of 
array which allows for specific mathematical operations, like slicing.

\subsection{Iterators} %subsection 1.1.12
\textbf{Iterators} are generalizations of \textbf{pointers}. They allow for access to the 
elements of collections. Every collection, regardless of its type, can be accessed using iterators. 
There are five types of iterators to distinguish:
\begin{itemize}
    \item input iterator
    \item output iterator
    \item forward iterator
    \item bidirectional iterator
    \item random access iterator
\end{itemize}
\textbf{C++ iterators are not defined in any manner of specific type (like a class)}. Instead, 
they’re defined by their behavior-supported operations. Every type of iterator has its own set of operations, 
which must be supported for a particular type to be called an iterator. This is the only constraint.

\subsection{Utilities} %subsection 1.1.13
The last department of \textbf{the Standard Template Library} is called \textbf{Utilities}. 
As the name suggests, it contains some \textbf{tools}, for example, for \textbf{date/time manipulation}, 
some supporting types like \textbf{pair}. The most important part is \textbf{the functional objects} 
(and functions) library. These objects are simple algorithms, defined for easy common tasks. 
For example, less(), which is used during the usage of the sort algorithm (and many more cases). 
The functional object library was designed to complement containers and algorithms in STL branches, 
but most of them can also be used in other applications.


\subsection{assign()} %subsection 1.4.7
\begin{methodinfo}
  {assign}
  {template <class InputIterator>
      void assign ( InputIterator first, InputIterator last ); 
      void assign ( size_type n, const T& u );}
  {\texttt{first, last}: the input iterators which provide a collection of input elements. 
  The \texttt{assign} method will copy all the elements from this range, including \texttt{first} and 
  excluding \texttt{last}. Because \texttt{first} and \texttt{last} are of the \texttt{InputIterator} type, 
  virtually any type of iterator can be used in the call

  \texttt{n}: the number of times the value will be copied to fill the container.

  \texttt{u}: the value to be copied}
  {None}
  {This method assigns new values to an already existing container. The whole old content of the 
  container is dropped and deleted.

  The new content is provided by one of two means: by either the iterator’s range or value, 
  which will fill a certain number of elements.

  In both cases, the new elements which are stored inside the collections are obtained by 
  copying the source values. As a result of the dropping of all the old elements, it’s possible 
  to use a source container of a different size to the target one. After the assignment, they will 
  both have the same size.}
\end{methodinfo}

\textcolor{green}{File name: 1.4.7.cpp}
\lstinputlisting[language=C++]{Chapter1/codes/1.4.7.cpp}


\subsection{push\_back() and pop\_back()} %subsection 1.4.12
\begin{methodinfo}
  {pop_back()}
  {void push_back (const T& x);}
  {\texttt{x}: the value which will be used to create a new element (by copying) inside a container.}
  {None}
  {The function \texttt{push\_back()} adds a new value to a container. The value is added at the end 
  (the back) of the container, and increases the size of the container by one. Different types of containers 
  react differently:}
  \begin{itemize}
    \item if a \texttt{vector} has enough capacity, the item is just added to it, no reallocation is performed, 
      and all obtained iterators remain valid
    \item if there is not enough capacity left, a reallocation is performed, which invalidates all iterators
    \item In the case of \texttt{deque}, all iterators are invalidated
    \item For a \texttt{list} container, all iterators are left unaffected.
  \end{itemize}
\end{methodinfo}

\begin{methodinfo}
  {pop_back}
  {void pop_back();}
  {None}
  {None}
  {This function removes an element from the tail of the container. Basically, it is the opposite 
  method to \texttt{push\_back()}. During element removal, its destructor is called, and the container 
  size is reduced by one. In the case of a vector call, this method invalidates all iterators, 
  pointers and references referring to the removed element.

  It’s also worth noticing that \texttt{pop\_back()} only removes the value, and the value is 
  not returned. This method cannot be used as the \texttt{l-value}. To obtain a value, you should 
  use \texttt{back() first}.}
\end{methodinfo}

\textcolor{green}{File name: 1.4.12.cpp}
\lstinputlisting[language=C++]{Chapter1/codes/1.4.12.cpp}


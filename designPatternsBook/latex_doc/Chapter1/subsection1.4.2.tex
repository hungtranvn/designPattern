
\subsection{empty() and resize()} %subsection 1.4.2
\begin{methodinfo}
  {empty}
  {bool empty () const;}
  {None}
  {The method returns true if a container is empty and false otherwise.}
  {This method is used to indicate whether a container is empty or not. You should use this method 
  instead of calling \inlinecode{C++}{size()} to check if the list container is empty. Using calling 
  \inlinecode{C++}{size()} for a list might result in linear time performance for some STL implementations, 
  instead of constant ones.}
\end{methodinfo}

\begin{methodinfo}
  {resize}
  {void resize (size_type sz, T c = T() );}
  {\\\texttt{sz}: the new size of a container\\
  \texttt{c}: the value to copy in order to add elements into a container when the new size (sz) 
  is greater than the old size}
  {None}
  {This method changes the current size of a container, either by causing it to grow or shrink. 
  The new size is provided by the parameter \texttt{sz}.

  If the new size is greater than the old size, new elements are added to the container. 
  Those new elements are created by copying parameter c, or, if c is not provided, by copying 
  the default value for a particular type of element. The number of elements added is expressed 
  by a simple formula: new size minus old size.

  If the new size is smaller than the old size, the collection will shrink. All elements between 
  the new size and the old size will cease to exist – which might cause their destructors to be called.}
\end{methodinfo}

\textbf{Example}: In the example, we use the \texttt{empty()} method to check whenever containers are 
eligible for resizing. You must remember that the \inlinecode{C++}{resize()} method sets the current 
size of a container to the exact value provided by its first argument. This argument \texttt{sz} is not 
a delta, but the exact size to be set. Another thing to remember is that \texttt{sz} is \texttt{unsigned}, 
so providing a negative value might cause the container not to shrink, but rather to grow close to its 
limit – \inlinecode{C++}{max_size()}.

\textcolor{green}{File name: 1.4.2.cpp}
\lstinputlisting[language=C++]{Chapter1/codes/1.4.2.cpp}


\subsection{splice() – list only} %subsection 1.4.14
\begin{methodinfo}
  {splice}
  {void splice ( iterator position, list<T,Allocator>& x ); 
  void splice ( iterator position, list<T,Allocator>& x, iterator i ); 
  void splice ( iterator position, list<T,Allocator>& x, iterator first, iterator last );}
  {\texttt{position} – the position in the calling list where the elements will be inserted;\\
  \texttt{x} – the list from which the elements will be moved to the calling list\\
  \texttt{i} – the iterator to a single element from the source list, which will be 
      moved to the calling list\\
  \texttt{first, last} – the iterators which define the range of elements to be moved from 
    the source list to the destination. The range includes \texttt{first} and excludes \texttt{last}.}
  {None}
  {This method moves elements from a list specified as parameter \texttt{x}, and inserts them into 
  the list container which calls the method. The target list size increases by the number of elements 
  moved, while the source list size decreases accordingly. There are three versions of this method. A 
  method which moves the whole content of the source container. A method which moves one, and only one, 
  element specified by the iterator.A method which moves a range of elements specified by the iterators.
  In all cases, there’s no object destruction or construction involved during the call.}
\end{methodinfo}

\textcolor{green}{File name: 1.4.14.cpp}
\lstinputlisting[language=C++]{Chapter1/codes/1.4.14.cpp}

%*************************************section1.5*****************************************
%
%
\section{Container adaptors} %section 1.5
\subsection{Container adaptors} %subsection 1.5.1
In STL terminology, a container adaptor is a class which uses an STL container in order to provide 
\textbf{other functionality}. In our case, these adaptors use sequential collections to \textbf{create 
additional data structures}, like \textbf{stacks} or \textbf{queues}. The STL provides three types of container 
adaptors:
\begin{itemize}
    \item stack;
    \item queue;
    \item priority queue.
\end{itemize}
In the table, you can find all the methods provided by container adaptors. As you can see, their subset is 
rather \textbf{limited in comparison to standard containers}, but their names should mostly be familiar to you. 
Most of these methods work as a simple proxy, and call the methods of the underlying containers.

In order for adaptors to work, you need to \textbf{provide} them with a \textbf{proper object container}. 
This can be done during their initialization – \textbf{the adaptor constructor} allows for that. You can 
choose which container will be used, or rely on a default one, which is different for every adaptor class. 
Each adaptor class requires a slightly different interface and functionality from its underlying collection.

Generally, \textbf{only sequential containers can be used}, but there are other limits related to specific 
adaptors; we’ll describe those limits a little later. You don’t have to use STL classes as underlying container 
types, though. As long as a container provides the interface required by the container adaptor to work, it 
can be any class.

\subsection{stack} %section 1.5.1
\begin{methodinfo}
  {<stack>}
  {template < class T, class Container = deque<T> > class stack;}
  {\texttt{T}: the type of the elements stored in the stack\\
  \texttt{Container}: the type of the underlying storage container}
  {None}
  {None}
\end{methodinfo}
\textbf{The stack class is just an STL implementation of a stack data structure – the LIFO (last in first out) 
concept}. In such a container, you can only add and remove elements, at its one end, which is usually called 
the top. The top always points to the last inserted element, and only this element can be removed from the stack.

A stack requires the following interface from its internal container:
\begin{itemize}
    \item \inlinecode{C++}{back()}
    \item \inlinecode{C++}{push_back()}
    \item \inlinecode{C++}{pop_back()} 
\end{itemize}
Therefore \textbf{all sequential containers (vector,  deque, list)} can be used as underlying storage. 
If an internal container is not specified, a \texttt{deque} is used.

\subsection{Stack object initialization} %subsection 1.5.2
\begin{methodinfo}
  {constructor}
  {explicit stack( const Container\& cont = Container() );}
  {\texttt{cont}: a container provided to serve as internal storage. Its type must be the same as the type defined
  in the stack template.\\
  \texttt{other}: an already existing stack which is used as a template to instantiate the current object. 
  Its internal container must be of the same type as the object being created.}
  {None}
  {First, the constructor creates a stack using the internal container provided. The container does not have 
  to be empty; there might be objects already inside. However, \texttt{cont} needs to be of the same type as 
  specified in the stack object declaration – the Container template parameter.
  
  The second constructor is just a copy constructor. Again, the source stack and the stack being created 
  must use internal storage of the same type.}
\end{methodinfo}
The example shows the correct way to create stack objects.

\textcolor{green}{File name: 1.5.2.cpp}
\lstinputlisting[language=C++]{Chapter1/codes/1.5.2.cpp}

\subsection{Stack initialization – the wrong way} %subsection 1.5.3
1. Not allowed - iterator constructor \\
2. Not allowed - copy constructor source and target stack object using different storage containers \\
3. Not allowed - initialization using predefined container – using different storage object than declared

\textcolor{green}{File name: 1.5.3.cpp}
\lstinputlisting[language=C++]{Chapter1/codes/1.5.3.cpp}

\subsection{Stack – assignment operator} %subsection 1.5.4
\begin{methodinfo}
  {assignment operator =}
  {stack<T, Container>& operator=( const stack<T,Container>& other );}
  {\texttt{other: }the source stack object, its contents are copied and stored in the target stack. 
  Source and target objects must use the same type of internal container.}
  {A reference to a calling object (\*this)}
  {The assignment operator copies elements from other and places them in the calling object (l-value). 
  As in the case of the copy constructor, the source and the target stack objects must both be created 
  using \textbf{the same type of internal storage}.}
\end{methodinfo}

\textcolor{green}{File name: 1.5.4.cpp}
\lstinputlisting[language=C++]{Chapter1/codes/1.5.4.cpp}

\subsection{Stack – destructor, empty() and size()} %subsection 1.5.5
\begin{methodinfo}
  {destructor}
  {\~stack();}
  {None}
  {None}
  {Destroys a \texttt{stack} object. The \textbf{destructor} of a stack calls the destructors (if applicable) 
  of all objects stored inside the stack, and \textbf{deallocates} all used storage.}
\end{methodinfo}
\begin{methodinfo}
  {empty}
  {bool empty ( ) const;}
  {None}
  {\texttt{true} if the stack size is 0, and \texttt{false} otherwise.}
  {This method tests if the stack size is 0, which means the stack is empty. The function is just a proxy 
  which calls the method of the same name in the underlying container.}
\end{methodinfo}
\begin{methodinfo}
  {size}
  {size_type size ( ) const;}
  {None}
  {The number of elements currently stored inside the stack.}
  {Returns the number of elements stored in a stack. The function is just a proxy which calls the method 
  of the same name in the underlying container.}
\end{methodinfo}

\textcolor{green}{File name: 1.5.5.cpp}
\lstinputlisting[language=C++]{Chapter1/codes/1.5.5.cpp}

\subsection{Stack – top()} %subsection 1.5.6
\begin{methodinfo}
  {top}
  {value_type& top ( ); \\
  const value_type& top ( ) const;}
  {None}
  {A reference to the element at the top of the stack.}
  {This function returns a reference to the top-most element of the stack. Since the stack is 
  a \textbf{LIFO} container, it also means it returns a reference to the last pushed element.
  \texttt{top()} is a proxy method, which means it calls the \texttt{back()} method of an underlying container. 
  The member type \inlinecode{C++}{value_type} is defined as the type of the elements from the underlying container.}
\end{methodinfo}

\begin{methodinfo}
  {push}
  {void push ( const T& x ); \\
  const value_type& top ( ) const;}
  {\texttt{x} – the value to be \texttt{pushed} (copied) onto the top of the stack; \\
    \texttt{T} – this is the first template parameter – the type of the elements stored inside the container.}
  {A reference to the element at the top of the stack.}
  {The \texttt{push()} method adds a new value onto the top of the stack. The new value lies on top of 
  the previously pushed value. This effectively increases the size of the stack by one. The new element 
  is created by coping parameter \texttt{x}. This method is a proxy, which means it calls the 
  \inlinecode{C++}{push_back()} method in the underlying storage container.}
\end{methodinfo}
\begin{methodinfo}
  {pop}
  {void pop ( );}
  {None}
  {None}
  {The \texttt{pop()} method removes the top-most element from the stack, and therefore reduces 
  its size by one. The element placed under the current top-most element becomes the new top of the stack.

  It’s important to remember that the \texttt{pop()} call does not return the removed element. 
  If you want to store the value of the top of the stack, call the \texttt{top()} method. This method 
  is a proxy, which means it calls the \inlinecode{C++}{pop_back()} method in the underlying storage container.}
\end{methodinfo}

\textcolor{green}{File name: 1.5.6.cpp}
\lstinputlisting[language=C++]{Chapter1/codes/1.5.6.cpp}

\subsection{Queue class} %subsection 1.5.7
\begin{methodinfo}
  {<queue>}
  {template  < class T, class Container = deque<T> > class queue;}
  {\texttt{T} – the type of the elements stored in the queue;\\
  \texttt{Container} – the type of underlying storage container.}
  {None}
  {\texttt{queue} is a container which provides FIFO (first in first out) functionality. 
  In the STL, it’s implemented as a container adaptor, therefore it requires some other 
  container to work as storage space. As for the FIFO concept, the element can be added (
  pushed) to one end of the queue (the back) and removed (popped) from the other (the front).

  In order for the queue to work, it requires a storage object with the following interface:

  \inlinecode{C++}{front()}
  
  \inlinecode{C++}{back()}
  
  \inlinecode{C++}{push_back()}
  
  \inlinecode{C++}{pop_front()}
  
  Therefore, the STL containers \texttt{deque} and \texttt{list} can be used. Of course, any collection 
  which makes use of these methods will be appropriate. If the type of the underlying container is not 
  specified explicitly, \texttt{deque} is used (check the class queue signature).}
\end{methodinfo}

\subsection{Queue – initialization} %subsection 1.5.8
Most of the list methods are just proxy methods, calling the method in the underlying container.
\begin{methodinfo}
  {constructor}
  {explicit queue( const Container& cont = Container() );\\
   queue( const queue& other );}
  {\\\texttt{cont} – the container provided to serve as internal storage. Its type must be the same as 
  the type declared in the queue template; \\
  \texttt{other} – the already existing queue, which is used as a template to instantiate the current 
  object; its internal container must be of the same type as the object being created.}
  {None}
  {First, the constructor creates the \texttt{queue} using the container provided. The container does 
  not have to be empty; there might be objects already inside. An important fact is that this container 
  must be of the same type as the internal type specified in the \texttt{queue} object declaration – 
  the Container template parameter.\\
  Second, the constructor is a copy constructor. Again, the source \texttt{queue} object and the queue 
  being created must use the same type of internal storage.}
\end{methodinfo}

\begin{methodinfo}
  {destructor}
  {\~queue();}
  {None}
  {None}
  {Destroys the \texttt{queue} object. The queue destructor calls the destructors (if applicable) of all 
  objects stored inside the \texttt{queue} and deallocates all used storage.}
\end{methodinfo}
Here is the correct way to create \texttt{queue} objects.

\textcolor{green}{File name: 1.5.8.cpp}
\lstinputlisting[language=C++]{Chapter1/codes/1.5.8.cpp}

\subsection{Queue initialization – the wrong way} %subsection 1.5.9
1. Not allowed - iterator constructor\\
2. Not allowed - copy constructor source and target stack object using different storage containers\\
3. Not allowed - initialization using predefined container - using different storage object than declared

\textcolor{green}{File name: 1.5.9.cpp}
\lstinputlisting[language=C++]{Chapter1/codes/1.5.9.cpp}

\subsection{Queue – assignment operator} %subsection 1.5.10
\begin{methodinfo}
  {assignment operator =}
  {queue<T, Container>& operator=( const queue<T,Container>& other );}
  {\texttt{other} – the source queue object, whose contents are copied and stored in a target queue. 
  The source and target objects must use the same type of internal container.}
  {A reference to a calling object (*this).}
  {The assignment operator copies the elements from other and places them in a calling object (l-value). 
  As in the case of the copy constructor, the source and target objects must be created using the same 
  type of internal storage.}
\end{methodinfo}

\textcolor{green}{File name: 1.5.10.cpp}
\lstinputlisting[language=C++]{Chapter1/codes/1.5.10.cpp}

\subsection{Queue – empty() and size() methods} %subsection 1.5.11
\begin{methodinfo}
  {empty}
  {bool empty ( ) const;}
  {None}
  {true if the \texttt{queue} size is 0, and false otherwise.}
  {This method tests if the \texttt{queue} size is 0, which means that the queue is empty. The function 
  is just a proxy which calls the method of the same name in the underlying container.}
\end{methodinfo}
\begin{methodinfo}
  {size}
  {size_type size ( ) const;}
  {None}
  {The number of elements currently stored inside a \texttt{queue}.}
  {Returns the number of elements stored in the \texttt{queue} internal storage.  The function is just a 
  proxy which calls the method of the same name in the underlying container.}
\end{methodinfo}

\textcolor{green}{File name: 1.5.11.cpp}
\lstinputlisting[language=C++]{Chapter1/codes/1.5.11.cpp}

\subsection{Queue – front(), back(), push() and pop()} %subsection 1.5.12
\begin{methodinfo}
  {front}
  {value_type& front ( );\\
  const value_type& front ( ) const;}
  {None}
  {A reference to the element at the front of a \texttt{queue}.}
  {This function returns a reference to the front-most element of a queue. This element is the one which 
  has been in the \texttt{queue} the longest, and only this element can be removed (popped) from the \texttt{queue}. 
  \texttt{front()} is a proxy method, which means it calls the \texttt{front()} method of the underlying container. 
  The member type \inlinecode{C++}{value_type} is defined as the type of the elements from the underlying container.}
\end{methodinfo}
\begin{methodinfo}
  {back}
  {value_type& back ( );\\
  const value_type& back ( ) const;}
  {None}
  {A reference to the last element in the \texttt{queue}}
  {This function returns a reference to the last element of the \texttt{queue} – in other words, 
  to the last inserted (pushed) element. 

  \texttt{back()} is a proxy method, which means it calls the method of the same name in the 
  underlying container. The member type \inlinecode{C++}{value_type} is defined as the type of 
  the elements from the underlying container – the \texttt{T} parameter of the \texttt{queue} template. 
  The last element of the \texttt{queue} cannot be removed.}
\end{methodinfo}
\begin{methodinfo}
  {push}
  {void push ( const T& x );}
  {\texttt{x} – the value to be inserted (pushed) at the end of the queue;\\
    \texttt{T} – this is the first template parameter – the type of the elements stored inside the container.}
  {None}
  {The \texttt{push()} method adds a new element to the \texttt{queue}. The new value is placed after 
  the previously pushed one, and becomes the last element of the \texttt{queue}. This effectively increases 
  the size of the \texttt{queue} by one. The new element is created by performing a copy of parameter \texttt{x}. 
  This method is a proxy, which means it calls the \inlinecode{C++}{push_back()} method in the underlying 
  storage container.}
\end{methodinfo}
\begin{methodinfo}
  {pop}
  {void pop ( );}
  {None}
  {None}
  {The \texttt{pop()} method removes the first element from the \texttt{queue} and therefore reduces 
  its size by one. If the \texttt{queue} is not empty, the next element (previously second in the sequence) 
  becomes the new front of the \texttt{queue}.

  It’s important to remember that the \texttt{pop()} call doesn’t return the removed element. If you want 
  to retrieve the value of the front of the \texttt{queue}, call the \texttt{front()} method instead. 
  This method is a proxy, which means it calls the \inlinecode{C++}{pop_front()} method in the underlying 
  storage container.}
\end{methodinfo}

\textcolor{green}{File name: 1.5.12.cpp}
\lstinputlisting[language=C++]{Chapter1/codes/1.5.12.cpp}

\subsection{Priority queue} %subsection 1.5.13
\begin{methodinfo}
  {<queue>}
  {template < class T, class Container = vector<T>,\\
  class Compare = less<typename Container::value_type> >\\ 
  class priority_queue;}
  {\\\texttt{T} – the type of the elements stored inside the \inlinecode{C++}{priority_queue}\\
    \texttt{Container} – the internal storage provider\\
    \texttt{Compare} – the comparator used for ascertaining the strict weak ordering inside the 
    \inlinecode{C++}{priority_queue};\\
    \inlinecode{C++}{Container::value_type} – the type of the elements stored inside the 
    \inlinecode{C++}{priority_queue} – it holds the same meaning as \texttt{T}.}
  {None}
  {\inlinecode{C++}{priority_queue} is a data structure in which the greatest element is always the first one, 
  due to some predefined strict weak ordering condition. The STL \inlinecode{C++}{priority_queue} is just 
  an implementation of that principle.

  In order for \inlinecode{C++}{priority_queue} to function, it requires a container to provide internal storage. 
  This container must offer the following functionality and methods: the elements must be accessible through 
  the random access iterator; \inlinecode{C++}{front(); push_back(); pop_back();}

  Therefore, only \texttt{vector} and \texttt{deque} of the STL containers are applicable. But again, 
  any container supporting the required functionality will be feasible; a programmer is not limited to 
  just the STL containers. If the internal storage type is not specified, a \texttt{vector} is used.

  \inlinecode{C++}{priority_queue} maintains its order using a comparator. The comparator is a function 
  which takes two parameters and returns true if the first parameter is lower than the second due to some 
  strict weak ordering. It can be implemented as either a functional object, or a standalone method. 
  This comparator is used when a new element is inserted (pushed) into the \texttt{queue}, or removed 
  (popped) from the \texttt{queue}, to guarantee that the element on top of it is always the greatest. 
  If no comparator is specified during the instantiation of the \inlinecode{C++}{priority_queue} – less, 
  the STL predicate is used.

  The ordered sequence of elements is achieved by using the \texttt{heap} algorithm. 
  \inlinecode{C++}{priority_queue} uses the following methods (algorithms): 
  \inlinecode{C++}{make_heap(), push_heap(), pop_heap()}.}
\end{methodinfo}

\subsection{Priority queue – initialization } %subsection 1.5.14
\begin{methodinfo}
  {constructor}
  {explicit priority_queue ( const Compare& x = Compare(),\\
  const Container& y = Container() );\\
  priority_queue( const priority_queue& other );}
  {\texttt{x} – the comparator object used to ensure strict weak ordering of the \texttt{queue} – 
  must be of the same type as declared in the template parameters;\\
  \texttt{y} – the container provided to serve as internal storage. Its type must be the same as the 
  type defined in the \inlinecode{C++}{priority_queue} template declaration;\\
  \texttt{other} – the already existing \inlinecode{C++}{priority_queue}, which is used as a template 
  to instantiate the current object; its internal container must be of the same type as the object 
  being created. Also, both objects must use the same comparator.}
  {None}
  {First, the constructor creates a \inlinecode{C++}{priority_queue} using the internal container provided, 
  and compares the predicate. The container does not have to be empty; there might be objects already inside. 
  However, the cont needs to be of the same type as specified in the \inlinecode{C++}{priority_queue} object 
  declaration – the Container template parameter. If objects \texttt{x} and \texttt{y} are provided, their 
  copies are created; in other cases, the container and comparator are just initialized. After that, the 
  constructor calls \inlinecode{C++}{make_heap()}.

  Second, the constructor is just a copy constructor. It creates a new object by coping the existing one. 
  Both objects must be equal in terms of their template definition – the same type of container and comparator 
  predicates.}
\end{methodinfo}

\textcolor{green}{File name: 1.5.14.cpp}
\lstinputlisting[language=C++]{Chapter1/codes/1.5.14.cpp}

\subsection{Priority queue – initialization} %subsection 1.5.15
Priority queue initialization – the wrong way
\begin{itemize}
  \item 1. using external storage object only
  \item 2. using forbidden container type as internal storage
  \item 3. providing comparator but not container type
  \item 4. using different comparator object than declared - warning, but the comparator object type 
    is deducted from constructor parameter
\end{itemize}

\textcolor{green}{File name: 1.5.15.cpp}
\lstinputlisting[language=C++]{Chapter1/codes/1.5.15.cpp}

\subsection{Priority queue – assignment operator} %subsection 1.5.16
\begin{methodinfo}
  {assignment operator =}
  {priority_queue<T, Container>&\\
  operator=( const priority_queue<T,Container>& other );}
  {\texttt{other} – the source \inlinecode{C++}{priority_queue} object, whose contents are copied and 
  stored in a target; \inlinecode{C++}{priority_queue}. The source and target objects must use the same 
  type of internal container.}
  {A reference to a calling object (*this).}
  {The assignment operator copies the elements from other, and places them in a calling object (l-value). 
  As in the case of the copy constructor, the source and target objects must be created using the same 
  type of internal storage.

  1. Correct - source and target stack are of the same type
  
  2a. Incorrect - target and source queue are not of the same type - internal storage
  
  2b. Incorrect - target and source queue are not using the same comparator}
\end{methodinfo}

\textcolor{green}{File name: 1.5.16.cpp}
\lstinputlisting[language=C++]{Chapter1/codes/1.5.16.cpp}

\subsection{Priority queue – empty() and size() methods} %subsection 1.5.17
\begin{methodinfo}
  {empty}
  {bool empty ( ) const;}
  {None}
  {\texttt{true} if the \inlinecode{C++}{priority_queue} size is 0, and \texttt{false} otherwise.}
  {This method tests if the \inlinecode{C++}{priority_queue} size is 0, which means the \texttt{queue} is empty. 
  The function is just a proxy which calls the method of the same name in the underlying container.}
\end{methodinfo}
\begin{methodinfo}
  {size}
  {size_type size ( ) const;}
  {None}
  {The number of elements currently stored inside a \inlinecode{C++}{priority_queue}}
  {Returns the number of elements stored in a \inlinecode{C++}{priority_queue}. The function is just 
  a proxy which calls the method of the same name in the underlying container.}
\end{methodinfo}

\textcolor{green}{File name: 1.5.17.cpp}
\lstinputlisting[language=C++]{Chapter1/codes/1.5.17.cpp}

\subsection{Priority queue – top(), push() and pop()} %subsection 1.5.18
\begin{methodinfo}
  {top}
  {const value_type& top ( ) const;}
  {None}
  {A constant reference to the element at the top of a \inlinecode{C++}{priority_queue}}
  {This function returns a reference to the top-most element of a \inlinecode{C++}{priority_queue}. 
  Because a \inlinecode{C++}{priority_queue} uses strict weak ordering, the top element is the greatest one. 
  It’s also worth noticing that the returned reference is a \texttt{const}. There’s no non-const version of 
  this method, as in a \texttt{stack}, for example. This is because of the ordered nature of the 
  \inlinecode{C++}{priority_queue}. A reference would allow for unnoticeable changes of element values, 
  and would disrupt the order of the \texttt{queue}.

  \texttt{top()} is a proxy method, which means it calls the \texttt{front()} method of the underlying container. 
  The member type \inlinecode{C++}{value_type} is defined as the type of the elements from the underlying container.}
\end{methodinfo}
\begin{methodinfo}
  {push}
  {void push ( const T& x );}
  {\texttt{x} - the value to be pushed (copied) into the \texttt{queue};\\
  \texttt{T} – this is the first template parameter – the type of the elements stored inside the container.}
  {None}
  {The method \texttt{push()} puts a new value into the \inlinecode{C++}{priority_queue}, and rearranges it 
  to keep the proper heap structure. This effectively increases the size of the adaptor by one. The location 
  of the newly inserted element is not determined beforehand; it depends on the values of the elements 
  already stored inside the \texttt{queue}. The new element is created by copying parameter \texttt{x}. 
  This method is a proxy, which means it calls the \inlinecode{C++}{push_back()} method in the underlying 
  storage container. Afterward, the \inlinecode{C++}{push_heap()} method is called.}
\end{methodinfo}
\begin{methodinfo}
  {pop}
  {void pop ( );}
  {None}
  {None}
  {The \texttt{pop()} method removes the top-most element from the \inlinecode{C++}{priority_queue}, 
  and therefore reduces its size by one, and the element destructor is called in the process. Also, 
  this method ensures that the new top element is the greatest one out of all the remaining elements. 
  It’s important to remember that the \texttt{pop()} call doesn’t return the removed element.

  If you want to retrieve the value of the top of the \texttt{queue}, call the \texttt{top()} method. 
  This method is a proxy, which means it calls the \inlinecode{C++}{pop_heap()} algorithm, and 
  then \inlinecode{C++}{pop_back()} in the underlying storage container.}
\end{methodinfo}

\textcolor{green}{File name: 1.5.18.cpp}
\lstinputlisting[language=C++]{Chapter1/codes/1.5.18.cpp}


\subsection{push\_front() and pop\_front() \texttt{list} and \texttt{deque} only} %subsection 1.4.13
\begin{methodinfo}
  {push_front}
  {void push_front(const T& x);}
  {\texttt{x} – the value which will be used to create a new element (by copying) inside a container.}
  {None}
  {The function \texttt{push\_front()} adds a new value to a container. The value is added at the 
  beginning (the front) of the container, and increases the size of the container by one. Different 
  types of containers react differently:
  \begin{itemize}
    \item in the case of \texttt{deque}, all iterators are invalidated
    \item for a \texttt{list} container, all iterators are left unaffected.
  \end{itemize}}
\end{methodinfo}

\begin{methodinfo}
  {pop_front}
  {void pop_front();}
  {None}
  {None}
  {This function removes an element from the beginning of the container. Basically, it’s 
  the opposite method to \texttt{push\_front()}. During element removal, its destructor is called, 
  and the container size is reduced by one. It’s also worth noticing that \texttt{pop\_front()} only 
  removes the value, and the value is not returned. This method cannot be used as the l-value. 
  To obtain a value, you should use \texttt{front()} first.}
\end{methodinfo}

\textcolor{green}{File name: 1.4.13.cpp}
\lstinputlisting[language=C++]{Chapter1/codes/1.4.13.cpp}


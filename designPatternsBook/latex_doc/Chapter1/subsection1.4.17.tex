
\subsection{merge() – list only} %subsection 1.4.17
\begin{methodinfo}
  {merge (list)}
  {void merge ( list<T,Allocator>& x ); 
  template <class Compare> 
    void merge ( list<T,Allocator>& x, Compare comp );}
  {\texttt{x} – the source list, whose elements are to be merged into the list calling the function;\\
  \texttt{comp} – the binary predicate used to compare the elements from the source and target lists in order 
  to ensure the proper sequence of elements in the target list.}
  {None}
  {This method performs a merge of two sorted list. In order to do so, two iterators are used: 
  one in the target (calling) list, and the second in the source list. The \texttt{merge()} method 
  compares the objects’ pointers with those iterators, and if the source object is less than the 
  target, it’s removed from the source and placed in the target list at the target iterator position.

  If the source object is greater, the insertion iterator advances and the procedure repeats. 
  This operation is repeated until the insertion iterator reaches the \texttt{end()} of the target list. 
  At that moment, if there are any elements left in the source list, they’re all moved to the target list, 
  and placed at the end.

  The second version of the method uses an external comparator in the form of a binary predicate 
  to perform a comparison between the elements from the target and the source. The predicate takes 
  as its first argument an object from the target list, and as the second argument an element from 
  the source list. It should return true if the target is less than the source, and false otherwise. 
  It should perform the strictly weak strict  ordering.

  During \texttt{merge()}, the objects are moved from the source to the target. Effectively, no object is created, 
  copied, or deleted. This function requires both lists to be sorted before it can be called.}
\end{methodinfo}

\textcolor{green}{File name: 1.4.17.cpp}
\lstinputlisting[language=C++]{Chapter1/codes/1.4.17.cpp}


\subsection{vector::capacity() - vector only} %subsection 1.4.3
\begin{methodinfo}
  {vector<T>::capacity}
  {size_type capacity () const;}
  {None}
  {The total amount of space for elements currently allocated to a particular vector}
  {This method is present in the vector class only.

  The \texttt{vector} class has two methods related to its size. The first one is \texttt{size}, 
  which you already know. The second one is capacity.

  \texttt{capacity} is the total number of slots inside a vector which are currently allocated. 
  Some of the slots might be used; some of them might be free.

  \texttt{capacity} is always greater than or equal to \texttt{size}. As we said earlier, 
  the \texttt{vector} occupies a \textbf{contiguous memory area}. This approach has some limitations. 
  When a new element is inserted into a \texttt{vector}, its size is increased and the whole structure needs 
  to be reallocated. In practice, such an approach isn’t very effective. In order to limit the number 
  of reallocations, the vector is equipped with a capacity capability.

  Each time a new element is put into the \texttt{vector}, it  first checks to see if there is enough capacity. 
  If not, the vector is reallocated, but the new capacity is usually larger than the old capacity + 1. 
  So, the next insertion/addition of an element will not require any reallocation of the vector. 
  This approach improves the performance of the vector.}
\end{methodinfo}

\textcolor{green}{File name: 1.4.3.cpp}
\lstinputlisting[language=C++]{Chapter1/codes/1.4.3.cpp}

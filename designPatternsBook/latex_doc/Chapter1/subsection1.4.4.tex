
\subsection{vector::reserve() vector only} %subsection 1.4.4
\begin{methodinfo}
  {vector<T>::reserve}
  {void reserve (size_type n);}
  {\texttt{n}: the minimum value of capacity to be requested}
  {None}
  {This method allocates additional space for elements inside a vector. If the newly requested capacity 
  \texttt{n} is greater than the current capacity, reallocation is enforced. The new effective capacity 
  will be at least as large as the one requested.

  If reallocation happens, all iterators are invalidated. If a requested capacity is lower than the 
  current capacity, the method will do nothing.

  Calling this method will never affect the elements already placed in the vector. This method is a 
  way to prepare the vector to accept a certain number of elements. Reserving space beforehand 
  eliminates the need for reallocation after each insertion.}
\end{methodinfo}
\textcolor{green}{File name: 1.4.4.cpp}
\lstinputlisting[language=C++]{Chapter1/codes/1.4.4.cpp}

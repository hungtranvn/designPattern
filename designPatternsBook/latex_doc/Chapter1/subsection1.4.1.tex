
%**************************************section 1.4**************************************
\section{Operation} %section 1.4

\subsection{size and max\_size} %subsection 1.4.1
\begin{methodinfo}
  {size}
  {size_type size () const;}
  {None}
  {The number of elements which are currently stored inside a collection.}
  {This method returns the number of elements which are currently stored inside a container. 
  The size will change each time an element is added to or removed from the container.}
\end{methodinfo}

\begin{methodinfo}
  {max_size}
  {size_type max_size () const;}
  {None}
  {The maximum number of elements which can be held inside a container.}
  {The method returns the maximum physical capacity of a container. This value might depend on 
  the STL library implementation or an operating system, and will always be constant in the same environment.}
\end{methodinfo}

\textbf{Example}: The example shows the basic usage of the \inlinecode{C++}{size()} and 
\inlinecode{C++}{max_size()} methods. As you can see, the size changes after inserting and removing 
elements from the container. On the other hand, the maximum size remains constant.

\textcolor{green}{File name: 1.4.1.cpp}
\lstinputlisting[language=C++]{Chapter1/codes/1.4.1.cpp}

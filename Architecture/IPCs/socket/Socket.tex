% hungtran
\documentclass[13pt]{beamer}
%
% Choose how your presentation looks.
%
% For more themes, color themes and font themes, see:
% http://deic.uab.es/~iblanes/beamer_gallery/index_by_theme.html
%
\mode<presentation>
{
  \usetheme{CambridgeUS}     % or try Darmstadt, Madrid, Warsaw, ...
  \usecolortheme{beaver} % or try albatross, beaver, crane, ...
  \usefonttheme{default}  % or try serif, structurebold, ...
  \setbeamertemplate{navigation symbols}{}
  \setbeamertemplate{caption}[numbered]
} 

\usepackage[english]{babel}
\usepackage[utf8x]{inputenc}
\usepackage{xcolor}
\usepackage{multicol}
\usepackage{tikz}
\usepackage{tikz-uml}
\tikzumlset{font=\footnotesize\ttfamily, class width=6ex}
\usepackage{hyperref}

\usepackage{listings}
\definecolor{codegreen}{rgb}{0,0.6,0}
\definecolor{codegray}{rgb}{0.5,0.5,0.5}
\definecolor{codepurple}{rgb}{0.58,0,0.82}
\definecolor{backcolour}{rgb}{0.95,0.95,0.92}

\lstdefinestyle{myCustomCppStyle}{
  language=C++,
  numbers=left,
  stepnumber=1,
  numbersep=9pt,
  tabsize=2,
  showspaces=false,
  showstringspaces=false
}

\lstset{basicstyle=\tiny,style=myCustomCppStyle}

\lstdefinestyle{mystyle}{
    backgroundcolor=\color{backcolour},   
    commentstyle=\color{codegreen},
    keywordstyle=\color{magenta},
    numberstyle=\tiny\color{codegray},
    stringstyle=\color{codepurple},
    basicstyle=\ttfamily\footnotesize,
    breakatwhitespace=false,         
    breaklines=true,                 
    captionpos=b,                    
    keepspaces=true,                 
    numbers=left,                    
    numbersep=5pt,                  
    showspaces=false,                
    showstringspaces=false,
    showtabs=false,                  
    tabsize=1
}

\lstset{style=mystyle}

\usepackage{graphicx}
\graphicspath{ {./images/} }

\usepackage{tikz}
\usetikzlibrary{decorations.text}
\usetikzlibrary{shapes.geometric, arrows, positioning, calc, matrix}

\tikzset{
  basic box/.style={
    shape=rectangle, rounded corners, align=center,
    draw=#1, fill=#1!25},
  header node/.style={
    Minimum Width=header nodes,
    font=\strut\Large\ttfamily,
    text depth=+0pt,
    fill=white, draw},
  header/.style={%
    inner ysep=+1.5em,
    append after command={
      \pgfextra{\let\TikZlastnode\tikzlastnode}
      node [header node] (header-\TikZlastnode) at (\TikZlastnode.north) {#1}
      node [span=(\TikZlastnode)(header-\TikZlastnode)] at (fit bounding box) (h-\TikZlastnode) {}
    }
  },
  hv/.style={to path={-|(\tikztotarget)\tikztonodes}},
  vh/.style={to path={|-(\tikztotarget)\tikztonodes}},
  fat blue line/.style={ultra thick, blue}
}

\definecolor{mygray}{RGB}{208,208,208}
\definecolor{mymagenta}{RGB}{226,0,116}
\newcommand*{\mytextstyle}{\sffamily\Large\bfseries\color{black!85}}
\newcommand{\arcarrow}[3]{%
   % inner radius, middle radius, outer radius, start angle,
   % end angle, tip protusion angle, options, text
   \pgfmathsetmacro{\rin}{1.7}
   \pgfmathsetmacro{\rmid}{2.2}
   \pgfmathsetmacro{\rout}{2.7}
   \pgfmathsetmacro{\astart}{#1}
   \pgfmathsetmacro{\aend}{#2}
   \pgfmathsetmacro{\atip}{5}
   \fill[mygray, very thick] (\astart+\atip:\rin)
                         arc (\astart+\atip:\aend:\rin)
      -- (\aend-\atip:\rmid)
      -- (\aend:\rout)   arc (\aend:\astart+\atip:\rout)
      -- (\astart:\rmid) -- cycle;
   \path[
      decoration = {
         text along path,
         text = {|\mytextstyle|#3},
         text align = {align = center},
         raise = -1.0ex
      },
      decorate
   ](\astart+\atip:\rmid) arc (\astart+\atip:\aend+\atip:\rmid);
}
\title[IPCs]{Socket}
\author{Hung Tran}
\institute{Fpt software}
\date{\today}

\begin{document}

\begin{frame}
  \titlepage
\end{frame}

% Uncomment these lines for an automatically generated outline.
\begin{frame}{Outline}
  \tableofcontents
\end{frame}

\section{Socket}

\begin{frame}{Socket}
	\begin{center}
	\textcolor{blue}{\textbf{A bidirectional gateway that communicates with different processes on the same machine or different machines.}}
	\textcolor{blue}{\textbf{A socket is a combination of an IP and a port number. A port number is a communication endpoint that connects with an external device.}}
	\textcolor{blue}{\textbf{Socket build client-server architechture systems.}}
	\textcolor{blue}{\textbf{The types of socket are differentiated based on data transfer mechanism.}}
	\end{center}
	\begin{itemize}
		\setlength\itemsep{1em}
		\item \textbf{Stream socket: } Transmission control protocol
		\item \textbf{Datagram socket: } User datagram protocol
		\item \textbf{Raw socket: } Use datagram socket
		\item \textbf{Domain socket: } Provide a medium to communicate with the processes of the same host system.
		\item \textbf{Internet domain socket: } Communicate with the other processes a vailable in remote system.
	\end{itemize}
\end{frame}

\section{IPC Over Network}

\begin{frame}{Communication style}
	\begin{itemize}
		\item How data should transfer over the network?
		\item Data is divided into small packets during Transmission.
		\item The small packets are grouped to make the commplete data in the receiver side.
	\end{itemize}
\end{frame}

\begin{frame}{Name space}
\begin{itemize}
\item Types of connection system
\end{itemize}
\end{frame}

\begin{frame}{Protocol}
	\begin{center}
	\textcolor{blue}{\textbf{A protocol is a set of rules and procedures to follow when two entities want to communicate with each other in a network.}}
	\textcolor{blue}{\textbf{A protocol consists of error recovery mechanisms and synchronization mechanisms.}
\end{center}
\end{frame}

\begin{frame}{API for socket programming}	
	\begin{itemize}
		\setlength\itemsep{1em}
		\item the <sys/socket.h> library implements socket programming.
	\end{itemize}
\end{frame}

\begin{frame}{OSI architecture model}	
	\begin{itemize}
		\setlength\itemsep{1em}
		\item A network medium is required for two connected computers. (ex. the internet)
    \item It required a network adapter or network interface controller (network card) when a computer wants to connect with the internet.
    \item Open system interconnect (OSI) was introduced to provide a standard for communication between a wide range of software and hardware devices.
    \item There are several different types of protocol: FTP, HTTP
	\end{itemize}
\end{frame}

\begin{frame}{TCP/IP model}	
	\begin{itemize}
		\setlength\itemsep{1em}
	\end{itemize}
\end{frame}

\begin{frame}{Client-server architecture}	
	\begin{itemize}
		\setlength\itemsep{1em}
	\end{itemize}
\end{frame}

\end{document}
